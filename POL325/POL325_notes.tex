\documentclass{article}

%settings
\usepackage[utf8]{inputenc}
\usepackage[explicit]{titlesec}
\usepackage{amsmath, amsfonts, amssymb, amsthm}
\usepackage{braket}
\usepackage[margin=1.0in]{geometry}
\usepackage{bbold}
\usepackage{fancyhdr}
\usepackage{fancyvrb}
\usepackage{graphicx}
\usepackage{float}
\usepackage{longtable}
\usepackage{array}
\usepackage{hyperref}
\pagestyle{fancy}
\fancyhead[L]{\leftmark}
\fancyhead[R]{\thepage}

\title{POL325: Contemporary Latin American Politics}
\author{Shamel Bhimani}
\date{Fall 2025}

\begin{document}

\maketitle

\tableofcontents
\newpage

\section{Development and its Alternatives}
    \subsection{Cristóbal Kay (2018) Modernization and Dependency Theory. From The Routledge Handbook of Latin American Development Julie Cupples, Marcel Palomino-Schalscha, and Manuel Prieto, eds. New York: Routledge, pp 15-28.}
    \subsubsection{Modernization Theory}
    \noindent \textbf{Historical Context:} Emerged in the North (
1950s-1960s) during the Cold War. Arose post-WWII as decolonization
accelerated.\\

    \noindent \textbf{Core Idea:} Development is a transition from a
`traditional' to a `modern' society. It posits that underdeveloped countries
can `catch up' by replicating the experience of Western, developed countries.\\
    \begin{itemize}
        \item Change is seen as determined by internal factors
    \end{itemize}

    \noindent \textbf{Dualistic Typology:}
    \begin{itemize}
        \item \textbf{Traditional Socieites:} Simple, rural subsistence
        economy, family labour, primitive technology, low productivity.
        Characterized by partciularism, adscription, and collective
        orientation. `Traditionalism' itself is seen as a barrier due to
        fatalistic outlook.
        \item \textbf{Modern Societies:} Complex, industrial,
        market-oriented, wage labour, scientific technology, high
        productivity. Characterized by universalism, achievement
        orientation, self-orientation, upward social mobility, and rule of law.
    \end{itemize}

    \noindent \textbf{Key Theorists and Concepts:}
    \begin{itemize}
        \item \textbf{Walt W. Rostow (1960):} Proposed five universal stages
        of economic growth:
        \begin{itemize}
            \item [$1$.] The traditional society;
            \item[$2$.] The preconditions for take-off;
            \item[$3$.] \textbf{Take-off (the key turning point)};
            \item[$4$.] The drive to maturity;
            \item[$5$.] The age of mass-consumption.
        \end{itemize}
        \item \textbf{Samuel Huntington (1968):} Priotized
        \textbf{political order and stability} above other modernization
        goals, concerned that rapid social change could overwhelm political
        institutions. Critiqued mainstream MT for being too static, arguing
        all societies combine traditional and modern elements.
        \item \textbf{Other theorisits focused on:} Value changes (Moore),
        personality transformation (Lerner), psychological factors like the
        desire to achieve (McClelland), and entreprenurial spirit (Hagen).
    \end{itemize}

    \noindent \textbf{Modernization Theory in Latin America:}
    \begin{itemize}
        \item Largely absorbed uncritically by Latin American social
        scientists and policymakers.
        \item \textbf{Gino Germani (1981):} A notable exception who adapted
        MT. He argued that transition processes create conflicts and
        `asynchronies' as different social spehres change at different
        speeds. However, his work was criticized from a Marxist perspective
        for failing to address class and ethnic conflicts.
    \end{itemize}

    \noindent \textbf{Critique of Modernization Theory (by Andre Gunder Frank):}
    \begin{itemize}
        \item Empirically faulty and theoretically weak.
        \item \textbf{Main flaw:} Assumes underdevelopment is an original
        state and ignores how development and underdevelopment are part of a
        single process in the formation of the world capitalist system since
        the 15th century.
        \item Fails to account for the impact of colonialism and imperialism.
    \end{itemize}


\end{document}