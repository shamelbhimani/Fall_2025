\documentclass{article}

%settings
\usepackage[utf8]{inputenc}
\usepackage[explicit]{titlesec}
\usepackage{amsmath, amsfonts, amssymb, amsthm}
\usepackage{braket}
\usepackage[margin=1.0in]{geometry}
\usepackage{bbold}
\usepackage{fancyhdr}
\usepackage{fancyvrb}
\usepackage{graphicx}
\usepackage{float}
\usepackage{longtable}
\usepackage{array}
\usepackage{hyperref}
\pagestyle{fancy}
\fancyhead[L]{\leftmark}
\fancyhead[R]{\thepage}

\title{POL325: Contemporary Latin American Politics}
\author{Shamel Bhimani}
\date{Fall 2025}

\begin{document}

\maketitle

\tableofcontents
\newpage

\section{Development and its Alternatives}
    \subsection{Cristóbal Kay (2018) Modernization and Dependency Theory. From The Routledge Handbook of Latin American Development Julie Cupples, Marcel Palomino-Schalscha, and Manuel Prieto, eds. New York: Routledge, pp 15-28.}
    \subsubsection{Modernization Theory}
    \noindent \textbf{Historical Context:} Emerged in the North (
1950s-1960s) during the Cold War. Arose post-WWII as decolonization
accelerated.\\

    \noindent \textbf{Core Idea:} Development is a transition from a
`traditional' to a `modern' society. It posits that underdeveloped countries
can `catch up' by replicating the experience of Western, developed countries.
    \begin{itemize}
        \item Change is seen as determined by internal factors
    \end{itemize}

    \noindent \textbf{Dualistic Typology:}
    \begin{itemize}
        \item \textbf{Traditional Socieites:} Simple, rural subsistence
        economy, family labour, primitive technology, low productivity.
        Characterized by partciularism, adscription, and collective
        orientation. `Traditionalism' itself is seen as a barrier due to
        fatalistic outlook.
        \item \textbf{Modern Societies:} Complex, industrial,
        market-oriented, wage labour, scientific technology, high
        productivity. Characterized by universalism, achievement
        orientation, self-orientation, upward social mobility, and rule of law.
    \end{itemize}

    \noindent \textbf{Key Theorists and Concepts:}
    \begin{itemize}
        \item \textbf{Walt W. Rostow (1960):} Proposed five universal stages
        of economic growth:
        \begin{itemize}
            \item [$1$.] The traditional society;
            \item[$2$.] The preconditions for take-off;
            \item[$3$.] \textbf{Take-off (the key turning point)};
            \item[$4$.] The drive to maturity;
            \item[$5$.] The age of mass-consumption.
        \end{itemize}
        \item \textbf{Samuel Huntington (1968):} Priotized
        \textbf{political order and stability} above other modernization
        goals, concerned that rapid social change could overwhelm political
        institutions. Critiqued mainstream MT for being too static, arguing
        all societies combine traditional and modern elements.
        \item \textbf{Other theorisits focused on:} Value changes (Moore),
        personality transformation (Lerner), psychological factors like the
        desire to achieve (McClelland), and entreprenurial spirit (Hagen).
    \end{itemize}

    \noindent \textbf{Modernization Theory in Latin America:}
    \begin{itemize}
        \item Largely absorbed uncritically by Latin American social
        scientists and policymakers.
        \item \textbf{Gino Germani (1981):} A notable exception who adapted
        MT. He argued that transition processes create conflicts and
        `asynchronies' as different social spehres change at different
        speeds. However, his work was criticized from a Marxist perspective
        for failing to address class and ethnic conflicts.
    \end{itemize}

    \noindent \textbf{Critique of Modernization Theory (by Andre Gunder Frank):}
    \begin{itemize}
        \item Empirically faulty and theoretically weak.
        \item \textbf{Main flaw:} Assumes underdevelopment is an original
        state and ignores how development and underdevelopment are part of a
        single process in the formation of the world capitalist system since
        the 15th century.
        \item Fails to account for the impact of colonialism and imperialism.
    \end{itemize}

    \subsubsection{Dependence Theory (DT)}
    \noindent \textbf{Historical Context:} Arose in Latin America in the
mid-1960s, challenging MT. Influenced by theories of imperialism and the
Latin American structuralist school (ECLAC).\\

    \noindent \textbf{Core Idea:} Underdevelopment is not an original state
but a \textbf{`conditioning situation'} where the economies of some
countries (the periphery) are conditioned by the development and expansion
of others (the center).
    \begin{itemize}
        \item Development and underdevelopment are seen as two faces of the
same historical process of global capitalism.
        \item It analyzes the link between external (global capitalism) and
        internal (class structure, politics) factors.
    \end{itemize}

    \noindent \textbf{Precursor: Raul Prebisch and ECLAC Structuralism:}
    \begin{itemize}
        \item Developed the \textbf{center-periphery paradigm}.
        \item \textbf{Prebisch-Singer Thesis:} Argued the international
        trade system benefits the center at the expense of the periphery due
        to the long-term deterioration of the periphery's terms of trade (
        prices of its primary commodity exports fall relative to the
        industrial goods it imports).
    \end{itemize}

    \noindent \textbf{Main Strands of Dependence Theory:}
    \begin{itemize}
        \item [$1$.] \textbf{Structuralis Strand:} Seeks to reform the
        capitalist system. Uses heterodox development theory concepts.
        \item[$2$.] \textbf{Marxist Strand:} Argues dependency can only be
        overcome by overthrowing capitalism adn transitioning to socialism.
        Relies on historical materialism and the labour theory of value.
    \end{itemize}

    \noindent \textbf{Key Structuralist Theorists and Concepts:}
    \begin{itemize}
        \item \textbf{Osvaldo Sunkel:} Focused on how transnational
        corporations (TNCs) deepen dependece and cause
        \textbf{`national disintegration'}. TNCs weaken the national
        bourgeoisie, fragment society, and shape public policy against the
        national interst.
        \item \textbf{Celso Furtado:} Analyzed
        \textbf{`dependent consumption patterns'}. The consumption habits of
        the rich, influenced by developed countries, create a wasteful,
        capital-intensive, and import-demanding industrial structure that
        perpetuates income concentration and underdevelopment.
        \item \textbf{Fernando Henrique Cardoso and Enzo Faletto:}
        \begin{itemize}
            \item Emphasized analyzing \textbf{`situations of dependency'}
            rather than a single theory, focusing on internal manifestations.
            \item Characterized the process as
            \textbf{`dependent development'}, rejecting stagnationist views
            and acknowledging that economic growth could occur within a
            dependency framework.
            \item Argued a \textbf{`new dependency'} emerged under
            corporatist-authoritarian states controlled by a militarized
            technocratic bureaucracy.
        \end{itemize}
    \end{itemize}

    \noindent \textbf{Key Marxist Theories and Concepts:}
    \begin{itemize}
        \item \textbf{Theotonio Dos Santos:} Identified a
        \textbf{`new character of dependency'} rooted in industrial and
        technological dependence. The lack of a domestic capital goods
        industry and indigenous technological capacity prevents dependent
        economies form being `articulated' and achieving autonomous development.
        \item \textbf{Ruy Mauro Marini:} Focused on \textbf{unequal exchange} (transfer of surplus value to dominate countries) and the resulting \textbf{over-exploitation of labour} in dependent countries to maintain profit rates. Also developed the concept of \textbf{sub-imperialism}, where a larger dependent country like Brazil under its military regime could engage in imperialist practices toward weaker neighbours to solve its own problems of insufficient internal demand.
        \item \textbf{Andre Gunder Frank:} Coined the phrase
        \textbf{``the development of underdevelopment}. Argued that the
        metropolis-satellite linkages constantly recreate underdevelopment.
        Contended that Latin America has been capitalist since the colonial
        conquest, challenging the prevailing feudal/semi-feudal
        characterization and the political strategies based on it. His
        thesis was famously critiqued by Ernesto Laclau for overemphasizing
        market circulation while neglecting relations of production.
    \end{itemize}

    \noindent \textbf{Decline and Legacy of Dependence Theory:} It's
influence waned with the economic crisis of the 1980s and the rise of
neoliberalism. Remebered as the
\textbf{first major challenge to the Eurocentric character of the social sciences} to achieve global influence. It inspired a new generation of scholars to think about development from the perspective of the South.

\subsection{Laura Zapata-Cantu and Fernando González (2021) Challenges for Innovation and
Sustainable Development in Latin America: The Significance of Institutions and
Human Capital. Sustainability.}

    \subsubsection{Innovation and Sustainable Development in Latin America: Core Challenges}

    \noindent \textbf{Context:} Sustainable development is a critical 21st
century challenge, compounded by the COVID-19 pandemic which has shifted
government and business priorities. Innovation is essential for
transitioning to a more sustainable world.\\

    \noindent \textbf{Regional Challenges:}
    \begin{itemize}
        \item \textbf{High social inequality and poverty} remain significant
        obstacles.
        \item The region has a history of
        \textbf{political instability and corruption}, which impacts
        institutional performance in innovation and sustainability.
        \item Economies are vulnerable to \textbf{commodity price volatility} and uncertainty.
        \item There is a \textbf{marginal contribution to global innovation}, often measured by patent registrations.
        \item Latin American countries face ongoing issues with nutrition,
        sanitation, quality education, and economic modernization.
    \end{itemize}

    \subsubsection{Theoretical Framework for Analysis}

    \noindent Two key theoretical perspectives are used to analyze
innovation and sustainable development in the region:
    \begin{itemize}
        \item [$1$.] \textbf{Dynamic Capabilities (DCV):} A framework
        focusing on a
        country's or firm's ability to adapt and transform in response to a
        changing environemnt. It is composed of three core capabilities:
        \begin{itemize}
            \item \textbf{Sensing:} Identifying opportunities and threats by
            scanning markets and technologies. THis aligns with the Global
            Innovation Index (GII) dimensions of \textit{Institutions} and \textit{Market Sophistication}.
            \item \textbf{Seizing:} Mobilizing resources to capture
            opportunities by investing in technology, human capital, and new
            business models. This aligns with the GII dimensions of
            \textit{Human capital and research, Infrastructure, and Business sophistication}.
            \item \textbf{Transforming:} Continuously reconfiguring assets
            and organizational structures to maintain the competitiveness.
            This aligns with the GII dimensions of
            \textit{Knowledge and technology outputs and creative outputs}.
        \end{itemize}
        \item[$2$.] \textbf{Mission-Oriented Policies:} Systemic public
        sector initiatives aimed at solving specific societal problems (
        e.g., climate change) by mobilizing innovation across multiple
        sectors and actors.
        \begin{itemize}
            \item These policies define a clear direction, foster
            collaboration between public and private sectors, and drive
            technological and systemic change.
            \item They require strong institutional support and the
            development of dynamic capabilities within government to lead
            change.
        \end{itemize}
    \end{itemize}

    \subsubsection{Analysis of Latin America's Performance}

    \noindent \textbf{Innovation (Based on Global Innovation Index 2020):}
    \begin{itemize}
        \item \textbf{Overall Performance:} The region contributes
        marginally to global innovation. The analysis covered 15 countries.
        \item \textbf{Performance by Dynamic Capability:}
        \begin{itemize}
            \item \textbf{Sensing (Weak):} The
            \textbf{institutions dimension received the lowest average score}, indicating issues with the political, regulatory, and business environment. Countries with better overall GII scores (e.g., Chile) performed better in this dimension.
            \item \textbf{Seizing (Mixed):} The
            \textbf{Business Sophistication dimension had the best average score}. However, many countries show a deficit in human capital development and RnD investment, despite having knowledge workers.
            \item \textbf{Transforming (Partial Success):} Some countries
            like Costa Rica and Mexico perform relatively well in
            transforming capabilities (creative and knowledge outputs) but
            lag in sensing and seizing, suggesting innovation occurs without
            fully considering environmental changes.
        \end{itemize}
        \item \textbf{Key Deficits:} To improve innovation, Latin America
        needs to strengthen institutions, increase investment in RnD,
        develop human capital.
    \end{itemize}

    \noindent
\textbf{Sustainable Development (Based on Sustainable Development Report 2020):}
    \begin{itemize}
        \item \textbf{Overall Performance:} Most Latin American countries
        rank in the middle portion globally. Chile, Costa Rica, Uruguay, and
        Ecuador are in the top 50 of 166 countries.
        \item \textbf{Regional Priorities (Highest-Ranked SDGs):}
        \begin{itemize}
            \item SDG 1 (No Poverty);
            \item SDG 4 (Quality Education);
            \item SDG 13 (Climate Action)
        \end{itemize}
        \item \textbf{Significant Challenges (Lowest-Ranked SDGs):}
        \begin{itemize}
            \item \textbf{SDG 9 (Industry, Innovation, and Infrastructure)}
            was the lowest-ranked goal in 10 countries.
            \item \textbf{SDG 10 (Reduced Inequality)} was the lowest-ranked
            in eight countries, indicating it is not a key priority for most
            governments in the region.
        \end{itemize}
    \end{itemize}

    \subsubsection{Proposed Roadmap and Policy Implications}\

    \noindent \textbf{Integrated Approach:} Effective strategy requires a
systemic approach that aligns strong
\textbf{institutions, mission-oriented policies, dynamic capabilities, and innovation}.\\

    \noindent \textbf{Role of Institutions:} Strong, inclusive institutions
are crucial for setting the direction through mission-oriented policies that
coordinate public and private actors. Weak institutions make it difficult to
address critical problems.\\

    \noindent \textbf{Role of Government (Public Sector):}
    \begin{itemize}
        \item Government should not just fix market failures but actively \textbf{co-create markets and ecosystems}.
        \item The public sector must take a leading role in translating
        societal challenges into concrete missions. This requires compotent
        public agencies and dynamic capabilities within the state.
    \end{itemize}

    \noindent \textbf{Human Capital:} Substantial investment in education
and human resource development is essential to build the skills required for
innovation.\\

    \noindent \textbf{Policy Focus:} Policies must encourage a shift from a
risk-management approach to one focused on seizing opportunities for
sustainable innovation.

    \subsection{Eduardo Gudynas (2013) Debates on Development and its Alternatives in Latin America: A Brief Heterodox Guide, in Beyond Development: Alternative Visions from Latin America, M. Lang and D. Mokrani, eds. Quito: Fundación Rosa Luxemburg, pp.15-39.}
    \subsubsection{The Conventional Concept of Development}

    \noindent \textbf{Definition and Origins:} Development is conventionally
understood as economic and social progress, often tied to growth,
modernization, and higher standards of living. Its modern usage in social
sciences solidified after World War II, establishing a division between
`developed' and `underdeveloped' nations.\\

    \noindent \textbf{Core Ideas (Mid-20th Century):}
    \begin{itemize}
        \item Development becasue almost synonymous with economic growth.
        \item It was viewed as a \textbf{linear process} where
        `underdeveloped' countries should emulate Western nations, as
        described in Rostow's stages of growth.
        \item Indicators like Gross Domestic Product (GDP) became primary
        targets.
        \item This model legitimized resources exploitation and the
        destruction of cultures and the environment in the name of progress.
    \end{itemize}

    \subsubsection{Early Critiques and `Development Alternatives'}

    \noindent These critiques question the \textit{methods} of devlopment
but often accept its core goal of material progress.\\

    \noindent \textbf{Dependence Theory (1960s-1970s):}
    \begin{itemize}
        \item Challenged the linear view, arguing that
        \textbf{underdevelopment is a consequence of development}, not a
        prior stage.
        \item Ascribed this condition to colonialism, imperialism, and
        asymmetrical power relations between the industrial `centre' and the
        resource-exporting `periphery'.
        \item While critical, this school still prioritized
        industrialization and economic growth.
    \end{itemize}

    \noindent \textbf{Ecological Limits to Growth (1970s):}
    \begin{itemize}
        \item The 1972 Club of Rome report, \textit{The Limits to Growth},
        argued that \textbf{perpetual economic growth is impossible} on a
        finite planet.
        \item This reporth was widely attacked, including by some Latin
        American intellectuals who saw it as a threat to modernization adn
        the use of the region's resources.
        \item A Latin American response,
        \textit{Catastropher or New Society? (1975)} argued problems were
        sociopolitical, not physical, and proposed a socialist model but
        still defended economic growth, albeit regulated.
    \end{itemize}

    \noindent \textbf{Human-Centered Approaches:}
    \begin{itemize}
        \item Sought to separate development from purely economic growth and
        focus on social dimensions.
        \item Key concepts include:
        \begin{itemize}
            \item \textbf{Another development:} Emphasized meeting basic
            needs, self-reliance, and endogeneity (defined within each society).
            \item \textbf{Human scale development (Max Neef):} Focused on
            people, not objects, and addressed poverty as a plural concept
            of unmet needs.
        \end{itemize}
    \end{itemize}

    \noindent \textbf{Sustainable Development:}
    \begin{itemize}
        \item Initially (1980), it meant using renewable resources without
        exceeding their renewal rates to meet human needs.
        \item The WCED's `Our Common Future' (1988) report offered a widely
        cited by polysemic definition that ultimately
        \textbf{reconciled sustainability with economic growth}, turning
        opposites into mutually dependent concepts.
        \item This was criticized for being a `contradiction in terms' as
        ``nothing physical can grow indefinitely''.
    \end{itemize}

    \subsubsection{The Ideology of Progress and the Post-Development Critique}

    \noindent This school of though critiques the foundational
\textit{ideas} behind development itself.\\

    \noindent \textbf{Development as Ideology:} The persistence of
development, despite its failures, suggests it functions as an
\textbf{ideology} -- the contemporary expression of the ideology of
\textbf{progress}. This ideology is shared across conventional liberal,
conservative, and socialist thought.\\

    \noindent \textbf{Post-Development:}
    \begin{itemize}
        \item A post-structuralist approach that deconstructs development
        as a \textbf{discourse} that shapes thought, institutions, and
        practices.
        \item It does not seek `another development' but rather examines its
        ecological foundations.
    \end{itemize}

    \noindent \textbf{Critique of Modernity:} Questioning development
requires questioning the project of \textbf{Modernity}, which is rooted in:
    \begin{itemize}
        \item A universalizing model based on European culture;
        \item A linear concept of history as progress from `backwardness';
        \item A sharp \textbf{duality separating society from Nature}, with
        Nature being an object for human use.
    \end{itemize}

    \noindent \textbf{Coloniality of Power:} Modernity was introduced to
Latin America via colonialism, imposing certain ideas about society,
knowledge, and history while excluding others (e.g., indigenous knowledge
systems).

    \subsubsection{Alternatives to Development}

    \noindent These frameworks seek to move beyond the ideology of progress
and the assumptions of Modernity.\\

    \noindent \textbf{Core Concepts:}
    \begin{itemize}
        \item They are \textbf{post-capitalist and post-socialist}, breaking
        with the shared ideology of progress.
        \item They challenge the society/Nature duality, often incorporating
        biocentric ethics (recognizing Nature's intrinsic view) and
        relational ontologies.
        \item They prioritize \textbf{well-being in a broad sense},
        including collective, spiritual, and ecological dimensions, over
        material accumulation.
        \item They require intercultural exchange and the inclusion of other
        knowledge systems, especially indigenous ones.
    \end{itemize}

    \noindent \textbf{Buen Vivir (`Good Living'):}
    \begin{itemize}
        \item An example of alterantive to development, drawing heavily on
        Andean indigenous knowledge (e.g., \textit{sumal kawasy, suma qamana}) as well as critical Western traditions like biocentrism and feminism.
        \item It breaks with anthropocentrism and linear progress to focus
        on community well-being in harmony with Nature.
        \item It can be understood as a political platform for building
        social orders free from the constraints of modernity.
    \end{itemize}

    \subsubsection{Contemporary Context: Progressive Governments and Neoextractivism}

    \noindent Since the late 1990s, many Latin American `progressive'
governments have rejected neoliberalism. However, in practice, they largely
adhere to a conventional development model based on
\textbf{economic growth, investment and exports}.\\

    \noindent This has led to \textbf{progressive neoextractivism}: the
intensive exploitation of natural resources (mining, oil), to fund social
programs. This model perpetuates environmental and social harm, denies
ecological limits, and reproduces core tenants of development myth
identified by Furtado decades ago.

    \section{Governance and Protest in the Twenty-First Century}

    \subsection{Rosalía Cortés (2009) Social Policy in Latin America in the Post-neoliberal Era. In: Jean Grugel and Pía Riggirozzi (eds) Governance after Neoliberalism in Latin America. Palgrave Macmillan, New York.}
    \subsubsection{Introduction to Social Policy Evolution in Latin America}

    \noindent \textbf{Focus:} Social security and labour law refors since
mid-1980s.\\

    \noindent \textbf{Guiding Argument:} Governing coalitions determine
national economic strategy, influencing service distribution and social
protection access.\\

    \noindent \textbf{Mechanisms:} State-directed redistribution via
institutions covering goods/services, social assistance, insurance, and
labour legislations. These shape social protection levels and economic
rights distributions.\\

    \noindent \textbf{Aim:} Analyze new social policies in an emergent
post-neoliberal era.\\

    \noindent \textbf{Context:} Regional social policy matrix reshaped by
economic liberalization and neoliberal growth paradigm in late 1980s-1990s.\\

    \noindent \textbf{Neoliberal Social Policy:} Result of new governing
coalitions incorporating business interests (local/global) and International
Financial Institutions (IFIs) demands.

    \subsubsection{Changing Paradigms of Social Policy}

    \noindent
\textbf{Import Substitution Industrialization (ISI) Eta (1940s-mid-1970s)}
    \begin{itemize}
        \item \textbf{Countries:} Larger Latin American countries (
        Argentina, Brazil, Chile, Mexico).
        \item \textbf{Alliances:} Strengthened political alliances between
        governing parties, employers, and trade unions.
        \item \textbf{Union Influence:} Trade unions influenced policy,
        securing social insurance benefits and income policies.
        \item \textbf{Social Policies:} Broad access in education and
        health with stable budgets and centralized implementation.
        \item \textbf{Problems:} Recurrent financing issues, cyclical
        downturns, inflation resolved by IFI-sponsored stabilization (
        wage/expenditure constraints). Led to stagnating investment (
        infrastructure, health, education) and deteriorating living
        standards (`lost decade' of the 1980s).
        \item \textbf{Neglect:} Social Policies and labour law reinforced
        government-union links, directed protection mainly to workforce,
        neglecting rural and poorest populations. This segmented provision
        created long-term difficulties for universal access to welfare and
        justice.
    \end{itemize}

    \noindent \textbf{End of ISI and Neoliberal Shift (1980s Onwards):}
    \begin{itemize}
        \item \textbf{Transition:} ISI ended in the 1980s, economic strategies
        shifted from import substitution to open economies.
        \item \textbf{Governing Coalitions:} New governments eliminated
        unions from policy debates, adopted IFI strategies, and accommodated
        local/global business groups.
        \item \textbf{Neoliberal Reforms:}
        \begin{itemize}
            \item Curtailment of public expenditure;
            \item Privatization fo social security;
            \item Introduction of labour regulation flexibilization
            \item \textbf{Official View:} Policies would protect the poor
            from economic transition impacts.
            \item \textbf{Reality:} Reinforced negative impacts of open
            markets on labour and living conditions.
        \end{itemize}
        \item \textbf{Business Dominance:} Business interest, supported by
        IFIs, pushed for reduced state economic regulation and social
        protection.
        \item \textbf{Corporatism:} In Argentina and Mexico, historic links
        between governing parties and labour helped contain union
        opposition, leading to corporatist agreements for reform and social
        security privatization.
        \item \textbf{Initial Outcomes:} Neoliberal policies and financial
        liberalization brought immediate price stabilization and short-lived
        growth (except Chile).
        \item \textbf{Negative Outcomes:} Open markets led to declining
        output, business closures, and growing unemployment. International
        financial crises (mid-1990s) exacerbated economic crisis.
        \item \textbf{Stagnation and Inequality:} Prolonged period of
        stagnation and increasing inequality (1998-2003) contributed to
        social unrest.
        \item \textbf{Social Movements:} Weakened trade unions meant social
        movements and civil society organizations represented marginalized
        groups, demanding better living standards and state intervention.
        \item \textbf{Political Shift:} Led to dramatic political changes;
        Left-leaning presidents emerged since early 2000s, reflecting
        demands for change.
        \item \textbf{Post-Neoliberal Era:} Policy mix and macroeconomic
        reshaping varies by country; attempts to combine social citizenship
        with market-led policies, or state-centered economy. No consolidated
        new paradigm, but gradual, tentative alternatives to neoliberalism.
    \end{itemize}

    \subsubsection{Economic and Social Reform Under Neoliberalism}

    \noindent \textbf{Paradigm Shift:} Economic strategies shifted from
import substitution to open economies, reflecting global trends and new
governing coalitions.\\

    \noindent \textbf{Governing Coalitions:} Eliminated trade unions from
policy debates, acquiesced to local/global business demands, adopted IFI
strategies.\\

    \noindent \textbf{Neoliberal Social Reform (Key Pillars):}
    \begin{itemize}
        \item \textbf{Curtailment of public expenditure};
        \item \textbf{Privatization of social security};
        \item \textbf{Flexibilization of labour regulations}.
    \end{itemize}

    \noindent \textbf{Official vs. Actual Impact:} Policies officially aimed
to protect the poor, but in fact reinforced negative impacts of open markets
on labour and living conditions.\\

    \noindent \textbf{Role of Business/IFIs:} Business interests (supported
by IFIs) gained dominance, pushing for reduced state regulation and social
protection.\\

    \noindent \textbf{Political Maneuvering:} In Argentina and Mexico,
corporatist agreements (historic links between parties and labour) helped
contain union opposition to reforms.\\

    \begin{itemize}
        \item [A.] \textbf{Social Security Reforms under Neoliberalism:}
        \begin{itemize}
            \item \textbf{Privatization:} Began around 1992 (Argentina,
            Bolivia, Columbia);
            \item \textbf{Shift in Responsibility:} Pensions becamse an
            individual responsibility, replacing shared risks with privately
            managed individual savings accounts.
            \item \textbf{Scheme Structure:} Typically maintained a public
            basic pensions system and introduced a mandatory, fully financed
            private scheme with fixed worked contributions and variable
            benefits.
            \item \textbf{Consequences:} Maintained or deepened earning
            inequalities and breached the social contract between generations.
            \item \textbf{Limited Initial Opposition:} Existing public
            schemes had limited coverage (excluding rural and informal
            workers), so privatization mainly affected formal sector
            workers. Public schemes wsere also seen as financially failing,
            fostering a `no-alternative' view aong the urban middle class.
            \item \textbf{IFI Influence:} IFIs favoured pension
            privatization; debate exists on whether they conditioned loans
            or indirectly influenced market-oriented models. Domestic
            support was crucial.
            \item \textbf{Country Examples:}
            \begin{itemize}
                \item \textbf{Argentina:} Introduced a \textbf{mixed system} (Reformed public basic pensions + fully funded second tier managed by unions/private banks) after intense negotiations, gaining union approval through economic benefits.
                \item \textbf{Mexico:} Initial social security (IMSS) for
                ISI-era workers faced bankruptcy by the 1980s, exacerbating
                poverty. Ministry of Finance pushed for privatization. A
                1992 scheme avoided structural change, but the 1995
                financial crisis led to a new 1997 scheme: increased
                retirement benefit threshold, closed the old scheme for new
                entrants (who joined private schemes), while maintaining
                privileged pensions for state officials. State's financing
                shared increased, eventually shifting all formal labour to
                the new system.
                \item \textbf{Brazil:} Did not follow the region's
                privatization pattern due to its segmented labour market.
                The 1988 Constitutions recognized universal social rights,
                but 1990s economic problems limited expansion. President
                Cardoso made small changes for private sector workers.
                Brazil implemented a `vertical mass expansion', extending
                targeted social assistance to uninsured workers, rather than
                universal access.
            \end{itemize}
            \item \textbf{Outcomes:} Low coverage persisted (2005: 65\%
            formal, 21\% informal workers contributed). Compliance fell.
            Private management became concentrated, with high administrative
            costs.
        \end{itemize}
        \item[B.] \textbf{Labour Market Reforms under Neoliberalism:}
        \begin{itemize}
            \item \textbf{Traditional Context:} Labour law was generally
            protective, guaranteeing employment regularity for formal
            workers (e.g., penalizing dismissals, limiting fixed-term
            contracts).
            \item \textbf{Neoliberal Reforms:} Aimed to flexibilize
            contracts, reduce termination penalties, facilitate outsourcing,
            and link wage increases to productivity.
            \item \textbf{Impact:} Negatively affected workers' living
            standards and increased insecurity.
            \item \textbf{Country Examples:}
            \begin{itemize}
                \item \textbf{Argentina:} Deep reforms, relaxing dismissal
                costs and transferring employment responsibilities to
                state/agencies/subcontractors, with some union support.
                \item \textbf{Mexico:} Adapted labour costs/discipline
                without new legislation. Agreements with trade unions (CTM)
                allowed flexible working, short-term contracts, and
                subcontracting. Fixed earnings meant real wages dropped 60\%
                over 20 years due to inflation.
                \item \textbf{Brazil:} 1988 Constitution increased worker
                protections. President Cardoso's 1998 legislation increased
                flexible working, limited individual rights, introduced
                part-time contracts, and removed fixed-iterm contract limits.
            \end{itemize}
            \item \textbf{Overall Outcomes:} Curtailed individual worker
            rights, expanded short-term contracts, leading to increased
            employment and income security. The number of unregistered
            workers (without basic labour rights) grew.
            \item \textbf{IFI Response:} IADB and World Bank programs
            provided loans for technical cooperation, training, and public
            works to manage reform fallout and embed labour market reforms.
        \end{itemize}
    \end{itemize}

    \subsubsection{Legacy of Economic Social Reform (Neoliberalism)}

    \noindent \textbf{Economic Performance:} Improved export performance,
controlled inflation, reduced fiscal deficits. However, output growth fell (
from 5.7\% to 3.5\% annually). Financial liberalization increased
vulnerability to foreign funds.\\

    \noindent \textbf{Shrinking State Protection:} State intervention
reoriented towards targeted assistance, reducing social protection and
enhancing the labour market's role in shaping living conditions.\\

    \noindent \textbf{Debate on Social Inequality:}
    \begin{itemize}
        \item \textbf{Walton (2004):} Neoliberal transformation `highly
        beneficial', increased growth without increasing social inequality.
        Attributed social problems to preexisting inequalities/ineffective
        institutions, nort market reforms. Noted mixed imapcts on
        inequality (Argentina up, Mexico down, Brazil improved).
        \item \textbf{Huber and Solt (2004):} Market reforms yielded
        disappointing results for social equity and democracy.
        \item \textbf{Contradictory Data:} Labour market and poverty data
        often contradict Walton's view.
        \begin{itemize}
            \item \textbf{Employment:} Unemployment rose (8.2\% to 10.5\%
            1990-2000), informal employment grew due to manufacturing
            crisis, privatization, and public sector cuts. Public sector
            jobs declined.
            \item \textbf{Job Quality/Security:} Deteriorated, leading to
            employment instability and income security.
            \item \textbf{Inequality:} Increased income concentration and
            widening inequality, with wage differentials between
            formal/informal sectors and declining informal sector incomes.
            Income share of poorest 40\% fell to 10\%.
            \item \textbf{Poverty:} While poverty dropped early 1990s, by
            1999 it was still higher than in the 1980s, and the income of
            the poor increased less than the non-poor.
        \end{itemize}
    \end{itemize}

    \noindent \textbf{Persistent Inequality:} High poverty levels attibuted
to persistent long-term income and asset inequality.\\

    \noindent \textbf{Social Services:} Changes reinforced barriers to
accessing social services. Administrative reforms insufficient; added
financial burden for the poor Middle/high-income groups shifted to private
providers.
    \begin{itemize}
        \item \textbf{Brazil (Exception):} Broadened social/political access
        to services to some extent, but mainly expanded social security
        within the formal sector, leaving informal workers ``out of the
        system''.
    \end{itemize}

    \noindent \textbf{Civil Society and Clientelism:} Claims that civil
society channeled demands of the poor, unlike political parties. However,
targeted local programs often reproduced exclusionary practices and
facilitated clientelism. Local political leaders controlled beneficiary
selection, reinforcing dependency. Not clear if it led to greater voice for
the poor.\\

    \noindent \textbf{Overall Conclusion of Neoliberal Era:} Economic
reforms transformed the social world, with social policy reinforcing (not
mediating) economic changes. Weakened labour movements and growing civil
society occurred amidst recession, falling incomes, and shrinking social
protection, leading ot policy crossroads in the early 21st century.

    \subsubsection{Emergence of New Economic and Social Strategies (Post-Neoliberal Era)}

    \noindent \textbf{Context (Post-2003):} Favourable global economy (
commodity prices, exports) improved fiscal accounts. Crisis of 1998-2003
eroded previous coalitions, leading to Left-leaning government winning
elections (Argentina, Brazil, Venezuela, Bolivia, Chile) on anti-neoliberal
platforms.\\

    \noindent \textbf{Uncertainty:} Unclear how much these new governments
will dismantle existing neoliberal programs or create entirely new strategies.\\

    \noindent \textbf{Country-Specific Trends:}
    \begin{itemize}
        \item \textbf{Argentina:}
        \begin{itemize}
            \item Began labour market re-regulation (2004).
            \item Reformed retirement system, encouraging return to public
            system, extended non-contributory pensions.
            \item Reoriented economic strategy (2002): currency devaluation
            boosted manufacturing/domestic market, urban employment.
            \item \textbf{Reinstated Collective Bargaining,} revived
            tripartite council for minimum wages (tripled by 2008).
            \item \textbf{Limitations:} Income policy priarily benefits
            formal waged workers, neglecting informal/precarious jobs.
            \item \textbf{Targeted Programs:} ``Plan Jefes y Jefas de Hogar
            Desocupados'' (cash transfers for household heads with conditons
            like schooling/vaccination) had broader but still insufficient
            coverage, with benefit values eroded by infaltions.
            \item \textbf{Emerging Model:} Based on a
            \textbf{dichotomy enhancing rights for formal sector workers while neglecting others}, intensifying the formal/informal labour divide. Shift from `poverty' to `labour issues' on policy agenda.
        \end{itemize}
        \item \textbf{Brazil:}
        \begin{itemize}
            \item \textbf{Lula da Silva:} Continued social security
            expansion for formal sector employees while directing expanding
            to poor families (``moderate reformism''). Unified
            public/private retirement schemes, reduced public employee benefits.
            \item \textbf{Targeted Programs:} Expanded conditional cash
            transfer programs (e.g., Bolsa Familia, which absorbed Fome Zero)
            . Bolsa Familia covered 11 million families by 2006, aiming to
            reduce poverty and incentivize school/health access.
            \item \textbf{Critique:} Coverage expansion may have political
            motivations.
            \item \textbf{Overall:} Combines neoliberal continuity in social
            security with new attention to targeted programs, increased
            minimum wages, and federal centralization of social policy. The
            overall policy thrust remains largely inherited.
        \end{itemize}
        \item \textbf{Mexico:}
        \begin{itemize}
            \item Social policy continued towards assistance for the
            vulnerable/poor, away from labour market support. Reflects
            collapse of ISI social coalition and declining importance of
            social security/labour rights.
            \item Prioritizes market role, with interventions restricted to
            cash transfer programs for the extremely poor.
            \item \textbf{Pioneer in CCTs:} Progresa (later Oportunidades)
            alleviated extreme poverty through conditional cash transfers (
            health checks, school attendance). Reached 24\% of the
            population by 2005, showing some success in secondary school
            enrollment and child nutrition.
            \item \textbf{Limitations:} Less effective for primary school
            attendance; doesn't solve long-term access to health/education (
            90\% of poor lacked healthcare in 2006).
            \item \textbf{Overall:} Fragmented social policies, with
            components from both pre- and post-neoliberal paradigms. Social
            security not fully privatized, education/health decentralized.
            Focus shifted from workers to extreme poverty.
        \end{itemize}
    \end{itemize}

    \subsubsection{Current Status of Social Policy in Latin America}

    \noindent \textbf{Cash Transfer Programs (CTPs):} Remain the
\textbf{paradigm of social policy} in Latin America, largely due to
governmental commitment to fiscal equilibrium.\\

    \noindent \textbf{Limitations of CTPS:}
    \begin{itemize}
        \item \textbf{Partial Coverage:} Only provide partial coverage and
        satisfy basic needs, especially without broader redistributive
        measures (e.g., tax reform). Exclude important segments of the
        poor (e.g., poorly paid workers).
        \item \textbf{Bypass Institutions:} Often bypass public social
        institutions, rarely building independent citizenship.
        \item \textbf{Social Control/Clientelism:} Can facilitate
        interaction between governments and the poor, potentially serving as
        a means fo social control or populist strategies by co-opting the
        poort and CSOs responsible for distribution. Clientelism arises when
        actors have discretion over benefit allocation.
    \end{itemize}

    \noindent \textbf{Overall Emerging Model:} A
\textbf{segmented and limited model}, divided between regulations/provisions
for formal workers and cash transfer programs for ``the poor''.\\

    \noindent \textbf{Future Challenges:} Universal social security
coverage, full access to quality employment, and improved education/health
are generally \textbf{not yet on the agenda}.\\

    \noindent \textbf{Regional Differences:} Argentina has begun to undo
elements of the neoliberal legacy, but this is less evident in Mexico,
Brazil and Chile

    \subsection{Marie-Christine Doran (2017) The Hidden Face of Violence in Latin America: Assessing the Criminalization of Protest in Comparative Perspective. Latin American Perspectives 44(216): 183-2016.}

    \subsubsection{Defining Criminalization of Protest}
    \noindent \textbf{Terminology:} Also referred to as criminalization of:
    \begin{itemize}
        \item Collective action/social protest;
        \item dissent;
        \item citizen participation;
        \item human rights defense.
    \end{itemize}

    \noindent \textbf{Nature:}
    \begin{itemize}
        \item Goes beyond mere penalization (imprisonment, new laws against
        activism).
        \item Invovles \textbf{intimidation and repression} based on
        antagonizing civil/political rights (essential to democracy) as
        threats to national security/interest.
        \item \textbf{Delegitimizes citizens} and paves the way for
        unpunished human rights violations.
        \item Analytically, it's a \textbf{collective trend} that takes into
        account power relations and critically analyzes the discursive
        battle over what is `peaceful' vs. `illegal' or `violent'.
        \item Includes enacting restrictive legislation, impunity for state
        officials, and refusal to investigate violations.
        \item Transforms peaceful, democratic actions into crimes subject to
        punishment.
        \item Legitimizes repression, turning community members into `public
        enemies' accused of violence, delinquency, or terrorism.
    \end{itemize}

    \textbf{Purpose:}
    \begin{itemize}
        \item \textbf{Obscures the political nature} of violence, presenting
        it as an objective fact necessary for security.
        \item A form of \textbf{depoliticization}, neutralizing political
        aspects and providing legal cover fore economic decisions.
        \item A indirect attack on the legitimacy of basic democratic conduct,
        leading to disregard for human life and gross human rights violations.
        \item Aims to \textbf{restrict democracy} and make it compatible
        with routinized violence against opponents, leading to
        ``violence-compatible democracies'' or ``violent pluralism''.
    \end{itemize}

    \subsubsection{Causes and Enabling Factors of Criminalization}

    \noindent Criminalization is rooted in:
    \begin{itemize}
        \item [$1$.] \textbf{Continuity with Authoritarianism:}
        \begin{itemize}
            \item \textbf{Identification of an internal threat} justifying
            unprecedented security policies.
            \item Militarization of police and increased military control
            over politics.
            \item Security policies in Latin America are continuous with
            antisubversion policies of authoritarian regimes.
        \end{itemize}
        \item[$2$.] \textbf{Influence of Low-Intensity Democracy:}
        \begin{itemize}
            \item \textbf{Reduction of democracy to a purely formal concept} (e.g., free elections, multi-party systems).
            \item \textbf{Delegitimization of all claim-based conflict}.
            \item This narrow view allows measures that render previously
            peaceful and legitimate practices illegal.
            \item Targets protests against natural-resource-based industry,
            educational rights, union rights, indigenous rights, and
            struggles against impunity.
        \end{itemize}
        \item[$3$.]
        \textbf{``Blaming the Victims'' under National Reconciliation:}
        \begin{itemize}
            \item Strategy to eliminate conflict and avoid political
            instability, or a new `breakdown of democracy'.
            \item State violence is attributed to the people's actions,
            making victims of dictatorship or state violence illegitimate.
            \item Justice for human rights violations is precluded, often
            justified as ``disadvantageous to democratic processes''.
            Impunity enshrined in reconciliation processes weakens the rule
            of law.
            \item Fear of social conflict and demands (especially for
            justice) has limited participation and led to regimes
            characterized as `post-dictatorships'.
        \end{itemize}
        \item[$4$.] \textbf{Intensification of Security Policies:}
        \begin{itemize}
            \item Latin America, identified as the most violent continent,
            has become a ``laboratory for security policies'' (war on
            terrorism, drug trafficking, crime).
            \item These policies create new fronts against social conflict,
            where any opposition is deemed a threat.
            \item The state and police adopt a new role: protecting
            citizens (seen as victims of themselves) from uncontrollable
            violence, justifying a strong, Hobbesian state.
            \item This disclosure reduces past state responsibility for
            violence and reasserts state control over a ``violence-prine
            society''. Militarization of society, as seen in Mexico, expands
            the military's role into civil society segments.
        \end{itemize}
    \end{itemize}

    \subsubsection{Case Studies: Mexico and Chile}

    \noindent The article compares Mexico and Chile, two countries with
different levels of civil violence, to illustrate criminalization.\\

\noindent \textbf{ A. Mexico: Beyond the War on Drugs}
\begin{itemize}
    \item \textbf{Context:} One of the most significant cases of
    criminalization since 2000, exacerbated by the ``War on Drugs''.
    \item \textbf{Scale of Violence/Repression:}
    \begin{itemize}
        \item Thousands of lawyers, activists, environmental
        defenders, and journalists imprisoned, tortured,
        assassinated, or forcibly disappeared.
        \item \textbf{300\% increase} in disappearances of activists
        and human rights defenders since 2012.
        \item \textbf{25,000 forced disappearences} in six years (
        2008-2014), surpassing records of Southern Cone
        dictatorships. These are concentrated in states with vibrant
        social organization like Oaxaca.
        \item Systematic and widespread torture (Mexico among top
        five countries globally in 2014).
        \item High rates of sexual violence against women protestors
        during arrests/interrogations (nearly 75\% of imprisoned women).
    \end{itemize}
    \item \textbf{State Involvement and Collusion:}
    \begin{itemize}
        \item Violence is often presented as solely due to drug traffickers,
        but this \textbf{underestimates the role of state forces} and their
        collusion with criminal networks.
        \item \textbf{Narcopastel Dynamic:} Drug traffickers invest dirty
        money in large projects, often involving natural resources, with the \textbf{complicity of political/judicial authorities}. This specifically targets environmental defenders.
        \item Direct collaboration between organized crime and
        police/military elements. Examples:
        \begin{itemize}
            \item \textbf{Padre Martin Octavio Garcia (2010):} Attacked,
            abducted, illegally detained, and tortured for organizing against
            a mine project in Oaxaca.
            \item \textbf{Cerezo Committee (2016):} Received death threats
            for defending human rights organizations.
            \item \textbf{Ayotzinapa Massacre (2014):} Exemplifies
            systematic persecution of social mobilizations.
        \end{itemize}
    \end{itemize}
    \item \textbf{Legislative and Constitutional Changes (2008):}
    \begin{itemize}
        \item Broadened role and discretionary powers of security forces;
        \item strengthened impunity for human rights violations by military
        and police;
        \item criminalized various forms of collective action and restricted
        prisoner rights.
        \item Legalized militarized police, allowing military courts to
        handle human rights violations by soldiers, upholding
        \textit{fuero militar} (military immunity). This was justified by
        `efficiency' in fighting drug trafficking, but had unexpected
        negative impacts on human rights.
        \item Established \textbf{preventive incommunicado detention} (\textit{casas de arraigo}) for up to 80 days for organized crime suspects, used to repress protest and social movements.
        \item Application of \textbf{military justice to civilians},
        denying due process.
        \item Criminalization of traditional protest practices, such as
        occupying political offices (\textit{planton}).
    \end{itemize}
    \item \textbf{Judiciary's Role:} The judiciary in Mexico
    \textbf{actively supports criminalization}, taking social conflicts into
    the judicial sphere and imprisoning activists.
    \item \textbf{Protection Mechanisms:} New measures like the Victims' Law
    and the Mechanism for the Protection of Journalists and Human Rights
    Defenders have been largely \textbf{ineffective} and serve as a
    ``strategy of simulation'' to legitimize the government externally.
\end{itemize}

    \noindent
\textbf{B. Chile: Authoritarian Legacy of the Pinochet Dictatorship}
    \begin{itemize}
        \item \textbf{Context:} Despite being a model of democratic
        transition with low civil violence, Chile exhibits troubling
        criminalization of human rights defense.
        \item \textbf{Key Legislation:} Primarily uses the
        \textbf{Antiterrorist Act (Law 18.314))}, inherited from the
        Pinochet dictatorship (1984) and strengthened in 1997.
        \begin{itemize}
            \item Views peaceful actions, protest organization, and land
            occupations as `terrorist acts'.
            \item Allows trials under military court provisions and
            sentences up to 30 years.
            \item \textbf{Mapuche indigenous activist} are
            disproportionately targeted and serve sentences for defending
            ancestral lands and the environment.
            \item Includes abroad definition of `threat' and `terrorist
            intent', even criminalizing hunger strikes by political prisoners.
        \end{itemize}
        \item \textbf{International Condemnation:} The
        \textbf{Inter-American Court of Human Rights (IACHR)} found the
        systematic application of the Antiterrorist Act to Mapuche actions
        inconsistent with the American Convention on Human Rights, violating
        the presumptions of innocence, equality, freedom of expression, and
        political rights.
        \item \textbf{Ongoing Repression:}
        \begin{itemize}
            \item Despite condemnations, the Antiterrorist Act is still
            applied, and efforts were made to broaden its provisions with
            the defeated Hinzpeter bill (2012).
            \item Beatings, torture, and forced disappearances of students
            and activists were documented following protests.
            \item \textbf{Stereotyping:} Media and authorities
            systematically portray indigenous people, especially the
            Mapuche, as violent, leading to disproportionate sentencing (
            e.g., a pregnant Mapuche woman sentenced for stealing a lunchbox
            based on ``presumed use of violence'').
            \item Repression of
            \textit{pobladores} (residents of marginal neighbourhoods)
            protesting housing issues, with threats to invoke the National
            Security Council and suspend constitutional guarantees by
            defining peaceful protests as `public calamity'.
        \end{itemize}
        \item \textbf{Judiciary's Role:} In contrast to Mexico, the Chilean
        judiciary has played a \textbf{progressive role} in countering
        violent practices against students. Adherence to ILO Convention 169 )
        right of indigenous peoples to be consulted) also offers potential
        for strengthening legal defense of indigenous rights.
    \end{itemize}

    \subsection{Adriana Piatti-Crocker (2021) Diffusion of NiUnaMenos in Latin America: Social Protests Amid a Pandemic. Journal of International Women’s Studies 22(12)}

    \subsubsection{\#NiUnaMenos Movement Overview}

    \noindent \textbf{Meaning:} \#NiUnaMenos translates to `Not One (woman)
less'.\\

    \noindent \textbf{Origin:} Convcieved in Argentina in 2015 to protest
misogynist violence.\\

    \noindent \textbf{Trigger Event:} Sparked by the May 2015 murder of
14-year-old Chiara Paez, whose body was found severely beaten and buried by
her boyfriend.\\

    \noindent \textbf{Founders:} A group of Argentine journalists, artists,
prominent feminists, and lawyers.\\

    \noindent \textbf{Main Goal:} To combat violence against women and
girls (VAWG).\\

    \noindent \textbf{Key Demands:}
    \begin{itemize}
        \item Effective implementation of Law 26, 485 (integral protection
        against VAWG).
        \item Publication of official statistics on femicide.
        \item Creation of shelters for women victims of violence.
        \item Protection of women's rights, including the legalization of
        abortion and comprehensive sex and gender education.
    \end{itemize}

    \noindent \textbf{Philosophical Approach:} Embraces a `feminism from
below' that is intersectional, transversal, and horizontal, engaging with
marginalized communities.\\

    \noindent \textbf{Impact:} Provided women across Latin America a
platform to demand greater gender equity and an end to misogynist violence.

    \subsubsection{Diffusion of NiUnaMenos}

    \noindent \textbf{Definition of Diffusion:} The process by which
institutions, practices, behaviours, or norms are transmitted among
individuals and/or social systems. The ``transfer in the same or similar
shape of forms and claims of contention across space or across sectors and
ideological divides.''\\

    \noindent \textbf{Diffusion Mechanism:}
    \begin{itemize}
        \item \textbf{Relational Models:} Involve interpersonal contact and
        communication between transmitters and adopters. Activists often
        used similar social networks.
        \item \textbf{Non-relational Models:} Emphasize channels of
        information diffusion not dependent on personal contact, such as the
        media and \textbf{digital platforms}.
    \end{itemize}

    \noindent \textbf{Role of Social Media:}
    \begin{itemize}
        \item \textbf{Platforms Used:} Facebook, Instagram, Twitter, and
        WhatsApp.
        \item \textbf{Impact:} Changed the depth and scope of protests,
        leading to an \textbf{unprecedented speed} in transmitting messages,
        strategies, identities, and goals.
        \item \textbf{Functionality:} Allowed large numbers of people
        worldwide to access and participate in protests through
        microblogging (e.g., Twitter) and social networking (e.g., Facebook)
        . Hashtags connected groups with meaning and content.
        \item \textbf{Rapid Spread:} The \#NiUnaMenos hashtag was globally
        trending on June 3, 2015, the day of the first protest. A Facebook
        page for the movement gained 130,000 likes early that month.
    \end{itemize}

    \noindent \textbf{Transitional Spread and Adaptation:} The movement
diffused rapidly across Latin America, replicated in major cities like Lima (
Peru), Quito (Ecuador), and Mexico City. Diffusion often involves
\textbf{adaptation (re-invention)} to local cultural or institutional
circumstances, while maintaining similar goals.
    \begin{itemize}
        \item \textbf{Example of Adapted Hashtags in Mexic (2016-2020):}
        \begin{itemize}
            \item \textbf{\#MiPrimerAcoso (My First Assault):} to share
            experiences of sexual assault.
            \item \textbf{\#SiMeMatan (If They Kill Me):} to protest
            authorities falsely claiming femicide committed suicide.
            \item \textbf{\#JuntasyOrganizadas (Together And Organized):}
            for marches and strikes.
        \end{itemize}
    \end{itemize}

\subsubsection{COVID-19 Pandemic's Impact on NiUnaMenos}

    \noindent \textbf{Contrasting Effects:}
    \begin{itemize}
        \item \textbf{Increased Urgency:} Measures to contain COVID-19
        exacerbated gender-based violence (termed the `second pandemic'),
        making \#NiUnaMenos demands more urgent.
        \item \textbf{Challenges to Mobilization:} Government policies to
        contain the virus (e.g., stay-at-home orders, bans on gatherings)
        restricted mass street demonstrations, which were a core strategy.
    \end{itemize}

    \noindent \textbf{Movement Adaptation:} Initially, women's groups became
more creative by organizing \textbf{virtual protests} to hold leaders
accountable. Later in 2020 and beyond, street protests gradually resumed as
domestic violence reached ``unprecedented levels of urgency'' and policy
inaction persisted.\\

    \noindent \textbf{Exacerbation of Violence Against Women and Girls (VAWG):}
    \begin{itemize}
        \item Mandatory lockdowns and stay-in-place orders confined women
        with abusers, removing previous escapes.
        \item Worldwide, domestic violence increased \textbf{10\% to 30\%}
        within the first weeks of lockdown.
        \item \textbf{Latin American Examples (2020):}
        \begin{itemize}
            \item Peru: Over \textbf{1,000 women disappeared} between March
            and June.
            \item Paraguay: Domestic violence cases rose by over
            \textbf{35\% per day} during lockdown.
            \item Columbia: \textbf{99 women murdered} (January-June);
            emergency hotline saw a \textbf{230\% increase} in domestic
            violence calls.
        \end{itemize}
    \end{itemize}

    \noindent \textbf{Governmental Responses and Their Efficacy:} Many Latin
American countries adopted COVID-19 specific measures (e.g., essential
service declarations for shelters, new hotlines, pharmacy reporting points).
However, despite these policies, domestic violence and femicide rates
remained high, and implementation was often ineffective. Street
demonstrations were often more effective and accessible for poor women with
limited or no access to virtual venues.

    \subsubsection{Policy Change and Remaining Challenges}

    \noindent \textbf{Increased Visibility:} \#NiUnaMenos significantly
increased in visibility of gender-based violence and femicide across the
region.\\

    \noindent \textbf{Existing Legislation:} Even before \#NiUnaMenos, 24 of
30 Latin American/Caribbean countries had legislation to protect victims of
domestic violence, and over half had laws against femicide.\\

    \noindent \textbf{Persistent High Rates:} Despite legislation, femicide
and domestic violence rates remain high and often underreported (e..g.,
Bolivia had the highest femicide rate in South America in 2019 at 2.1 per
100,000 women; Mexico had 1.5 per 100,000).\\

    \noindent \textbf{Need for Effective Action:} While digital platforms
are crucial for activism, sustained change requires that activism and
governmental policy translate into effective, operational action.\\

    \noindent \textbf{Root Causes:} The problem is deeply rooted in Latin
America due to pervasive sexism, discrimination, and machismo.


\section{The Changing Left-Right Divide and the Durability of Elite Power}

    \subsection{Thomas Chiasson-LaBel and Manel Larrabure (2019) Elite and
Popular Responses to a Left in Crisis. European Review of Latin American and Caribbean Studies 108: 87-107.}

    \subsubsection{Central Thesis: A Relational Approach to the Pink Tide's Crisis}

    \noindent The crisis of the `pink tide' (a wave of left-of-centre
governments from the early 2000s) is best understood through a
\textbf{relational approach}.\\

    \noindent This approach challenges \textbf{state-centred} perspectives
that overemphasize the autonomy of the state and the role of charismatic
leaders.\\

    \noindent It focuses on the \textbf{balance of class forces} by
analyzing the interactions between three key actors:
    \begin{itemize}
        \item Governments;
        \item Economic Elites;
        \item Popular Movements.
    \end{itemize}

    \noindent The central question is whether pink tide governments
successfully empowered popular sectors while undermining the political
capacity of economic elites.

    \subsubsection{Timeline of the Pink Tide's Decline}

    \noindent \textbf{Early Signs:} The 2009 coup in Honduras and the
impeachment of Fernando Lugo in Paraguay were initial indicators.\\

    \noindent \textbf{`Undeniable' right turn (2015-2016):}
    \begin{itemize}
        \item \textbf{Argentina (2015):} Muaricio Macri, a wealthy
        businessman, won the presidency.
        \item \textbf{Brazil (2016):} The impeachment of Dilma Rousseff set
        the stage for the rise of the far-right Jair Bolsonaro.
        \item \textbf{Venezuela (2015):} The opposition coalition won a
        majority in the National Assembly for the first time since 1998.
    \end{itemize}

    \noindent \textbf{Further Signs of Decline:}
    \begin{itemize}
        \item \textbf{Ecuador:} President Lenin Moreno shifted to a
        neoliberal path, splitting his own party.
        \item \textbf{Bolivia:} Evo Morales lost a 2016 referendum on term
        limits but ran for re-election anyway in 2019; protests over
        suspected electoral irregularities led to his ousting.
    \end{itemize}

    \subsubsection{Critique of Mainstream Analytical Frameworks}

    \noindent \textbf{State-Centred Perspectives:} These analyses assume the
state acts independently of social forces, focusing only on government
actions. They often exhibit a paternalistic tendency a `coloniality of
knowledge' by using the Global North as a model for comparison without
adequately integrating local context.\\

  \noindent \textbf{The Populism Thesis:} This popular framework divides the
left into a `good' institutional left (e.g., Brazil's PT) and a `bad'
populist left (e.g., Chavez, Correa, Morales).
    \begin{itemize}
        \item \textbf{Critique 1:} This thesis incorrectly portrays populist
        leaders as the \textit{cause} of weak institutions. A better
        perspective sees populism as the \textit{result} of a broken
        institutional order that fails to meet popular demands.
        \item \textbf{Critique 2:} The focus on charismatic leaders portrays
        popular sectors as passive followers, overlooking the vibrant and
        often autonomous social movements that were active during this period.
    \end{itemize}

    \subsubsection{The International Context and its Internal Mediation}

    \begin{itemize}
        \item [A.] \textbf{The Economic Context:}
        \begin{itemize}
            \item \textbf{Commodities Boom:} Driven by China's growth, a
            boom in commodity prices allowed Latin American economies to
            resist the 2008 global financial crisis.
            \item \textbf{`Re-primarization':} A key failure of all pink
            tide governments was their inability to diversify their
            economies. Instead, they deepened their reliance on primary
            resource extraction. This left them vulnerable when China's
            growth slowed and commodity prices fell after 2012-2014.
            \item \textbf{Internal Mediation:} The effects of the global
            economy were mediated locally. For example, Bolivia and Ecuador
            pursued counter-cyclical public investment, while Brazil adopted
            austerity, which was still not enough to satisfy local elites.
        \end{itemize}
        \item[B.] \textbf{The Geopolitical Context:}
        \begin{itemize}
            \item \textbf{`International Bonapartism':} The rise of the left
            was enabled by a geopolitical `window of opportunity' where US
            hegemony was diminished (due to its focus on Iraq and
            Afghanistan) and China had not yet become a dominant global power.
            \item \textbf{Alternative Regionalism:} This period allowed for
            the creation of new regional institutions (e.g.,
            \textbf{UNASUR, ALBA-TCP, CELAC}) that challenged US influence
            and provided stability for left-wing governments.
            \item \textbf{Decline of Alternative Regionalism:} These
            institutions were heavily dependent on the governments that
            created them an on Venezuelan oil rents. They did not survive
            the pink tide's decline, the end of the commodities boom, and
            the reassertion of US influence in the region.
        \end{itemize}
    \end{itemize}

    \subsubsection{Key Findings from the Relational Approach}

    \noindent The core argument is that the changing balance of power is
best understood by examining the relationships between the state, elites,
and popular sectors in specific national contexts.

    \noindent
\textbf{Conclusion 1: Left governments failed to durably undermine the political power of economic elites.}
    \begin{itemize}
        \item Elites developed strategies to block reforms, regain
        influence, and eventually recapture state power.
        \item Examples of Elite Resilience:
        \begin{itemize}
            \item \textbf{Argentina:} Elites engaged directly in politics by
            forming the PRO party, which won the 2015 election.
            \item \textbf{Ecuador:} Elites adapted to the Correa
            government's institutional changes, developed new means of
            influence, blocked reforms like an inheritance tax, and
            eventually gained access to the executive.
            \item \textbf{Bolivia:} Elites from Santa Cruz instigated
            protests and forced the Morales government into a `catastrophic
            standoff', which resulted in a less radical program and
            compromises with elite interests.
            \item \textbf{Chile:} The center-left government developed
            intimate ties with economic elites, who in turn created
            sophisticated `technologies of power' to ensure consent for
            extractivist projects.
        \end{itemize}
        \item
        \textbf{Conclusion 2: Interactions between governments and popular movements did not sufficiently strengthen popular sector capacities to counterbalance elites or radicalize change.}
        \begin{itemize}
            \item The relationship was complex, often involving simultaneous
            support and defiance, rather than a simple alliance.
            \item In many cases, left government
            \textbf{weakened autonomous social movements}, viewing them as
            threats to stability rather than as essential allies.
            \item Examples of weakened movements:
            \begin{itemize}
                \item \textbf{Ecuador:} The Correa government bypassed,
                repressed, and closed institutional spaces previously won by
                the Indigenous movement (CONAIE).
                \item \textbf{Brazil:} The PT government formed alliances
                with right-wing parties, disappointing its base and failing
                to cope with ne democratic demands that erupted in 2013.
                \item \textbf{Bolivia:} The government dismissed left-wing
                critiques from social movements with paternalistic labels
                like `NGOism, the infantile disease of rightism'.
            \end{itemize}
        \end{itemize}
    \end{itemize}

    \subsection{Luis Bonilla (2018) Captured Democracy: Government for the Few (Executive Summary). OXFAM International}

    \subsubsection{Core Concepts: Captured Democracy and State Capture}

    \noindent \textbf{State's Role in a Democracy:} To design public
policies that tackle poverty and inequality while enhancing citizens.\\

    \noindent \textbf{`Captured' State:} A state is `captured' when it
grants privileges to a few over the majority, and its policies redice or
limit the rights of citizens. This process involves the abusive influence of
an elite for its own interests over the public policy cycle.\\

    \noindent \textbf{Public Perception in LAC:} 75\% of the population
believes their government is run by powerful groups for their own benefit. 65\% report dissatisfaction with democracy.

    \subsubsection{The Vicious Cycle: Inequality, Capture, and Democracy}

    \noindent \textbf{Inequality as a Democratic Indicator:}  High
inequality is a significant indicator of a democracy's quality, as
democracies are based on the premise of equal rights for all.\\

    \noindent \textbf{Relational Cycle:}
    \begin{itemize}
        \item High concentration of wealth and power increases the elite's
        capacity to shape laws and policies to protect their privileges.
        \item This `state capture' leads to policies that reproduce inequality.
        \item To tackle capture, one must tackle inequality, and vice versa.
    \end{itemize}

    \noindent
\textbf{The Paradox: Increasing democracy is the only way to limit state capture}, which will increase equality. However, \textbf{increasing equality is also the only way to increase democracy}.\\

    \noindent \textbf{The Solution:} The imbalance can only be addressed by
ensuring \textbf{greater citizen participation and representation}.

    \subsubsection{Fiscal Policy and Its Impact in LAC}

    \noindent \textbf{Purpose:} Fiscal policy is the state's primary tool
for wealth redistribution.\\

    \noindent \textbf{Ineffectiveness in LAC:}
    \begin{itemize}
        \item Fiscal policy in LAC is a `missed opportunity' for reducing
        inequality.
        \item While OECD countries reduce market income inequality by 35\%
        through fiscal policy, the reduction in LAC is only \textbf{6\%}.
    \end{itemize}

    \noindent \textbf{Poverty Increase:}
    \begin{itemize}
        \item In six LAC countries, fiscal policy's net effect
        \textit{increased} the number of people living in poverty (Guatemala,
        Honduras, Nicaragua, Bolivia, the Dominican Republic, and El Salvador).
        \item The poorest people in six countries are
        \textbf{net contributors} to the fiscal system, rather than
        beneficiaries, due to the impact of indirect taxes like VAT. This
        reduces their access to basic human rights like health and education.
    \end{itemize}

    \subsubsection{Mechanisms of Fiscal Capture by Elites in LAC}

    \noindent Elites use various mechanisms to influence fiscal policies for
their own benefit. The most common are:
    \begin{itemize}
        \item [$1$.] \textbf{Media Campaigns (69\% of cases):}
        \begin{itemize}
            \item \textbf{Definition:} Controlling information to define the
            public debate agenda and shape public opinion. Media ownership
            is highly concentrated (`media feudalism').
            \item \textbf{Example:} In Paraguay (2012), a media company
            owned by a major shareholder in soya bean exporting company led
            a campaign that contributed to the ousting of a president who
            supported taxing soya products.
        \end{itemize}
        \item[$2$.] \textbf{Revolving Door (69\% of cases):}
        \begin{itemize}
            \item \textbf{Definition:} The unhindered movement of
            high-ranking individuals between the public and private sectors,
            leading to conflicts of interest.
            \item \textbf{Example:} In Argentina (2017), 40\% of top
            officials in the treasury ministry were former CEOs or managers
            from the private sector.
        \end{itemize}
        \item[$3$.]
        \textbf{Extraordinary Regulatory Procedures (62\% of cases):}
        \begin{itemize}
            \item \textbf{Definition:} Routinely using non-standard
            procedures (e.g., executive orders, fast-tracked bills) to
            bypass public deliberation and participation.
            \item \textbf{Example:} In Peru (2015), 62\% of public-private
            partnership (PPP) contracts has addenda added after being
            signed, suggesting contracts are won with non-viable proposals
            and then renegotiated.
        \end{itemize}
        \item[$4$.]
        \textbf{Exploiting Political/Electoral System Weaknesses (54\% of cases):}
        \begin{itemize}
            \item \textbf{Definition:} Using party funding and other
            electoral system features to secure fiscal privileges.
            \item \textbf{Example:} Brazilian company Odebrecht
            systematically funded political campaigns across LAC to secure
            lucrative public works contracts.
        \end{itemize}
        \item[$5$.] \textbf{Lobbying (46\% of cases):}
        \begin{itemize}
            \item \textbf{Definition:} A disproportionate capacity to lobby
            due to greater financial resources and access to informal
            networks (e.g., shared schools, social circles.)
            \item \textbf{Example:} In Chile (2014), a tax reform bill
            negotiated in private meetings between government officials and
            private sector representatives, outside of any formal lobbying law.
        \end{itemize}
        \item[$6$.] \textbf{Technical Smokescreen (31\% of cases):}
        \begin{itemize}
            \item \textbf{Definition:} Repackaging political decisions as
            highly technical issues to exclude the general public from the
            debate. Elites can also fund think tanks to present their
            interests as objective research.
            \item \textbf{Example:} In Peru (2014), a tax reform law was so
            technical that 74\% of the population reported not knowing about it.
        \end{itemize}
        \item[$7$.]
        \textbf{Exploiting Regulatory Frameworks for Participation (31\% of cases):}
        \begin{itemize}
            \item \textbf{Definition:} Creating or using regulations to
            ensure elite over-representation in decision-making bodies.
            \item \textbf{Example:} In the Dominican Republic (2014),
            committees on tax incentives for companies had no space for
            citizen participation and were dominated by private sector
            interests.
        \end{itemize}
        \item[$8$.] \textbf{Judicialization of Policy (23\% of cases):}
        \begin{itemize}
            \item \textbf{Definition:} Using the court system, particularly
            constitutional courts, to delay or block fiscal reforms that
            affect elite interests.
            \item \textbf{Example:} Guatemala's Constitutional Court
            received hundreds of petitions between 1996-2018 or to halt or
            weaken tax reforms, often benefiting business elites.
        \end{itemize}
        \item[$9$.] \textbf{Bribes and Influence Padding:}
        \begin{itemize}
            \item \textbf{Definition:} Illegal payments to influence policy
            and budget allocation.
            \item \textbf{Example:} Odebrecht admitted to paying \$780
            million in bribes across 10 LAC countries, which allowed the
            company to achieve profits of almost \$3 billion.
        \end{itemize}
        \item[$10$.] \textbf{Co-opting Marches and Protests:}
        \begin{itemize}
            \item \textbf{Definition:} Elites encouraging social
            mobilization to support measures that favour their own interests.
            \item \textbf{Examples:} In Ecuador (2015-16), business leaders
            led a campaign against an inheritance tax, mobilizing people who
            were unlikely to ever be affected by it.
        \end{itemize}
        \item[$11$.] \textbf{Opacity in Tax Havens:}
        \begin{itemize}
            \item \textbf{Definition:} Using banking secrecy ad tax havens
            to evade taxes and conceal illicit activities.
            \item \textbf{Example:} Odebrecht was linked to at least 17
            companies in tax havens, which were used to launder money and
            hide payments to politicians.
        \end{itemize}
    \end{itemize}

    \subsection{Lecture}
    \begin{itemize}
        \item The roster of elites hasn't changed much despite all events
        taking place in Latin America, and all the regime changes.
        \item Left/Right distinction as a matter of situating ideologies.
        \item Left considered an expansion of social rights;
        \item Right considered a constriction of social rights.
        \item Economically, left calls for regulation;
        \item Economically, the right calls for deregulation.
        privatization, austerity, and security.
        \item The distinction between left and right is
        \textit{relative depending on context} (very partial list of factors):
        \begin{itemize}
            \item State power;
            \item Previous experiences of the left or right in power;
            \item Allies and Affinities (IMF/International Financial
            Institutions (IFIs)?);
            \item Elite networks (regional/global powers);
            \item Cosmopolitanism vs. Tradition;
            \item Time (future/past, duration in position (government
            opposition));
            \item Relation/Reaction to Contestation.
        \end{itemize}
        \item Governments have increasingly seen political dissent and
        protest as dangerous and criminal as opposed to the liberal
        definition of a healthy society.
        \begin{itemize}
            \item This is especially prominent concerning national
            development projects being protested by land defenders, water
            defenders, indigenous defenders etc.
            \item A series of legal and discursive techniques are used,
            including anti-terrorism laws, restrictive legislation, impunity
            for official violators, refusal to investigate, `delinquency'
            and securitization of civil society, and stability and economic
            measures as supreme social values.
        \end{itemize}
        \item Seen in recent events in Chavez's Venezuela.
        \item A new reaction to socialist ideology in the context of the
        developing global order. This reaction is in line with traditional
        elite mechanisms and interests (establishing control over
        institutions and legal mechanisms to maintain elite status without
        legal or political recourse unlike political figures who experience
        this recourse).
        \begin{itemize}
            \item Return to Comparative Advantage, market fundamentals;
            \item Defense of traditions vs. accelerationism;
            \item populism -- mass mobilization;
            \item re-emergence of cold war framing;
            \item China, decline of the US, climate crisis and decarbonization;
            \item Persistence of elite power despite turbulence in the
            1990s-2000s.
        \end{itemize}
        \item The elite responses to the pink tide:
        \begin{itemize}
            \item Cooperation with governments in power and adapting to new
            realities;
            \item protests to changes and capitalizing to protests;
            \item Barry Cannon: `Coups, Smart Coups, and Elections'.
        \end{itemize}
    \end{itemize}

    \section{Bolivia: Continuities and Change during the MAS Era}

    \subsection{Gabriel Hetland (2019) Understanding Bolivia’s Nightmare. NACLA Report of the Americas https://nacla.org/news/2019/11/19/bolivia-morales-coup}

    \subsubsection{The Event: A Military Coup}

    \noindent The events in Bolivia culminating in President Evo Morales's
ouster constituted a coup.\\

    \noindent \textbf{Defining Factor:} The military demanded that Morales
step down.
    \begin{itemize}
        \item The military head's use of the word `suggest' is considered
        irrelevant; when the military intervenes to ask a president to
        leave, it is a coup.
        \item This occurred on November 10, and Morales resigned shortly after.
    \end{itemize}

    \subsubsection{Lead Up and Immediate Causes (2016-2019)}

    \begin{itemize}
        \item [A.] \textbf{Presidential Re-election Controversy}
        \begin{itemize}
            \item \textbf{2016 Referendum:} Morales narrowly lost a
            referendum on allowing indefinite presidential re-election.
            \begin{itemize}
                \item His supporters claimed a `dirty war' involving a media
                scandal about a `love child' may have influenced the result.
            \end{itemize}
            \item \textbf{2017 Court Decision:} An electoral court
            overturned the referendum's result, permitting Morales to run in
            2019.
            \item \textbf{Outcome:} This decision generated widespread
            dissent, particularly among urban middle classes.
        \end{itemize}
        \item[B.] \textbf{The October 20, 2019 Election}
        \begin{itemize}
            \item \textbf{Result:} Morales won the election in the first round.
            \item \textbf{Fraud Allegations:} The Organization of American
            States (OAS) led charges of fraud, stating on November 10 that
            it could not certify the result.
            \item \textbf{Counter-Evidence:} A report by the Center for
            Economic Policy and Research (CEPR) argues that OAS was biased
            and failed to present actual evidence of fraud.
        \end{itemize}
        \item[C.] \textbf{Protests and Escalation}
        \begin{itemize}
            \item \textbf{Initial Protests:} Large protests against Morales
            occurred in the weeks following the election, fueled by both the
            re-election isue and fraud allegations. They were initially led
            by urban middle classes.
            \item \textbf{Right-Wing Leadership: Luis Fernando Camacho}, a
            conservative businessman, led calls for Morales's resignation,
            pushing the protests to the political Right.
            \item \textbf{Final Events:}
            \begin{itemize}
                \item Police mutinies occurred on November 8 to 9.
                \item The military `suggested' Morales resign on November 10.
            \end{itemize}
        \end{itemize}
    \end{itemize}

    \subsubsection{Long-Term Contributing Factors}

    \noindent \textbf{The 2011 TIPNIS Conflict:} A plan to build a road
through the TIPNIS national park created divisions.
    \begin{itemize}
        \item Led to a split between Morales and some leftist figures and
        organizations.
        \item \textbf{Weakened Social Movements:} The government (MAS party)
        played a role in dividing popular organizations, weakening their
        capacity to mobilize in Morales's defense in 2019.
    \end{itemize}

    \subsubsection{The Post-Coup Regime: Jeanine Anez}

    \begin{itemize}
        \item [A.] \textbf{Nature of the Regime}
        \begin{itemize}
            \item Described as a \textbf{`far-right regime of terror'} and a \textbf{`dictatorship-in-embryo'}.
            \item Marked by systematic violation of political and human rights.
        \end{itemize}
        \item[B.] \textbf{Anez's Illegitimate Ascension}
        \begin{itemize}
            \item \textbf{Power Vacuum:} Assumed the presidency after the
            forced resignations of Morales, his VP, and the heads of both
            houses of Congress.
        \begin{itemize}
            \item Resignations were coerced via kidnapping of officials'
            relatives and arson.
        \end{itemize}
            \item \textbf{Unconstitutional:} As vice president of the
            Senate, she had no constitutional authority to assume the
            presidency.
        \item \textbf{Lack of Quorum:} Sworn in a nearly empty Senate, as
            MAS senators (two-thirds majority) were boycotting due to safety
            fears.
            \item \textbf{No Mandate:} Her political party received only 4\%
            of the vote in the October 20 election.
        \end{itemize}
        \item[C.] \textbf{Ideology and Policy}
        \begin{itemize}
            \item \textbf{Religious and Anti-Indigenous:} Anez and her
            mentor Camacho are described as fervently Christian and highly
            racist.
            \begin{itemize}
                \item Symbolic acts included bringing an oversized Bible to
                the palace and a pastor declaring ``Pachamama will never
                return''.
                \item Led to a wave of anti-Indigenous racism, including the
                public burning and police desecration of the
                \textbf{wiphala flag}, an official national symbol.
            \end{itemize}
            \item \textbf{Cabinet:} Her initial cabinet included no
            Indigenous ministers.
            \item \textbf{Foreign Policy:} Broke relations with Venezuela
            and Cuba.
        \end{itemize}
    \end{itemize}

    \subsubsection{State Repression and Human Rights Violations}
    \begin{itemize}
        \item [A.] \textbf{Violence Against Protestors}
        \begin{itemize}
            \item State security forces killed peaceful protestors with live
            bullets.
            \item \textbf{Sacaba Massacre (Nov 15):} Security forces killed
            nine nonviolent protestors.
            \item \textbf{Legal Impunity:} Anez issued a decree exempting
            armed forces from prosecution for use of force, effectively a \textbf{`license to kill'}.
            \item As of November 16th, at least 23 people had been killed
            and 715 injured in the violence.
        \end{itemize}
        \item[B.] \textbf{Political Persecution}
        \begin{itemize}
            \item The government threatened to detain MAS senators for \textbf{sedition and subversion} and prevent MAS from running in future elections.
            \item Anez's minister of government called a former MAS minister
            an `animal' he would `hunt down'.
            \item Threats were made against `seditious' journalists.
        \end{itemize}
    \end{itemize}

    \subsubsection{Resistance and International Reaction}

    \noindent \textbf{Domestic Resistance:}
    \begin{itemize}
        \item Massive Indigenous marches and protests erupted across the
        country against Anez's government and the repression.
        \item Mobilizations included MAS supporters and a broader base of
        popular sectors opposed to the right-wing seizure of power.
    \end{itemize}

    \noindent \textbf{International Reaction:}
    \begin{itemize}
        \item Condemnation from the Inter-American Commission on Human
        Rights and the UN High Commissioner for Human Rights.
        \item US Senator \textbf{Bernie Sanders} called the events a coup.
    \end{itemize}

    \noindent \textbf{Outlook:}
    \begin{itemize}
        \item The sources describe Bolivia's situation as a ``descent into a
        full-blown far-right military dictatorship''.
        \item The state's ferocious response suggests a ``long, uneven, and
        deadly struggle'' ahead.
    \end{itemize}

    \subsection{Linda Farthing and Benjamin Kohl (2014) The Land of Unintended Consequences. In Evo’s Bolivia: Continuity and Change. Austin, TX: University of Texas Press, Ch. 2.}

    \subsubsection{Geography and Demographics}

    \noindent \textbf{Physical Geography:}
    \begin{itemize}
        \item \textbf{Extremes:} Renowned for extremes in landscape,
        history, and resources.
        \item \textbf{Diverse Terrain:} Encompasses two Andean mountain
        ranges, the high-altitude \textbf{altiplano} (over 11,500 ft.),
        temperate valleys (~8,000 ft), Amazonian rainforests, wetlands, and
        the semidesert Chaco.
        \item \textbf{Geographic Superlatives:}
        \begin{itemize}
            \item World's highest navigable lake (Lake Titicaca).
            \item World's largest salt falt (Salar de Uyuni).
            \item World's second-largest wetlands (Pantanal).
        \end{itemize}
        \item \textbf{Administration:} Divided into nine departments.
    \end{itemize}

    \noindent \textbf{Population and Ethnicity:}
    \begin{itemize}
        \item \textbf{Density:} One of South America's lowest population
        densities.
        \item \textbf{Indigenous Majority:} Approximately
        \textbf{65\% of Bolivians self-identity as indigenous}, the highest
        percentage in the Americas.
        \begin{itemize}
            \item This identity is a fluid social category, not a racial one.
            \item The percentage claiming indigenous heritage grew steadily
            for two decades.
        \end{itemize}
        \item \textbf{Major Indigenous Groups:}
        \begin{itemize}
            \item \textbf{Highlands: Aymara} and \textbf{Quechua} speakers
            are the most populous groups.
            \begin{itemize}
                \item \textbf{Quechua:} A conglomeration of ethnicities
                merged by the Inka Empire.
                \item \textbf{Aymara:} Resisted both Inka and Spanish
                domination, retaining their language and culture.
            \end{itemize}
            \item \textbf{Lowlands (East):}
            \begin{itemize}
                \item \textbf{Chiquitanos} (~195,000): An ethnicity forged
                by Jesuits in the 17th-18th centuries; later subjected to
                force labour and debt peonage.
                \item \textbf{Guarani} (~130,000): Conquered in 1892, they
                have since faced continuous dispossession of their lands.
                \item \textbf{34 smaller groups} (~150,000 total):
                Devastated by disease, missions, and loss of territory.
            \end{itemize}
        \end{itemize}
        \item \textbf{Other Groups:}
        \begin{itemize}
            \item \textbf{Mestizos} (~25\%): People of mixed European and
            indigenous heritage, or urbanized indigenous people.
            \item \textbf{Criollos} (<10\%): Claim European heritage;
            historically the most economically and politically powerful group.
        \end{itemize}
    \end{itemize}

    \noindent \textbf{Socio-Economic Indicators (c. 2005):}
    \begin{itemize}
        \item \textbf{Urbanization:} Two-thirds of the population lives in
        cities.
        \item \textbf{Inequality:} One of the most unequal income
        distributions in the world; the wealthiest 10\% earned over 90 times
        more than the poorest 10\%.
        \item \textbf{Poverty:} 60\% of the population lived below the
        poverty line, with 37\% in extreme poverty. Rural poverty was nearly
        double that of urban areas.
        \item \textbf{Demographics:} Life expectancy was 68 years; over a
        third of the population was under 15.
    \end{itemize}

    \subsubsection{Indigenous Cultures and History}

    \noindent \textbf{Ancient Andean Civilizations:}
    \begin{itemize}
        \item \textbf{Tiwanaku} (300CE - 1000CE): Flourished near Lake
        Titicaca with complex hydrological systems; collapsed after a major
        drought.
        \item \textbf{Inka Empire} (15th Century): Dominated the region
        before the Spanish arrival through conquest and negotiation. They
        imposed the Quechua language guaranteed food security.
    \end{itemize}

    \noindent \textbf{Social and Cultural Structures:}
    \begin{itemize}
        \item \textbf{Ayllu:} A nested kinship system that continues to
        shape local governance.
        \item \textbf{Reciprocity and Collective Labour:} Practices like \textbf{ayni} (family labour exchange) and \textbf{mink'a} (community work) are common. Status is often based on contributions to the group's well being.
        \item \textbf{Cosmovision:} A worldview mixing animism with respect
        for nature.
        \begin{itemize}
            \item \textbf{Pachamama} (Earth Mother) is a central living deity.
            \item Time and space are seen as simultaneous reality, express
            in the word \textbf{pacha}.
        \end{itemize}
    \end{itemize}

    \noindent \textbf{Five Centuries of Globalization:}
    \begin{itemize}
        \item \textbf{Extractive Economy:} Bolivia's vast natural resource
        wealth (silver, tin, rubber, hydrocarbons) has historically profited
        global powers rather than local interests.
        \item \textbf{Spanish Conquest (1532):}
        \begin{itemize}
            \item \textbf{Potosi Silver Boom:} For over a century, Potosi's \textbf{Cerro Rico} (Rich Hill) produced half the world's silver, fueling Europe's industrial revolution.
            \item \textbf{Human Cost:} An estimated 1 to 4 million
            indigenous miners died. The indigenous population collapsed by
            75-90\% within 40 years due to disease and conquest.
        \end{itemize}
        \item \textbf{Resistance:} Indigenous resistance to domination has
        been constant, with major uprisings led by the Aymara and Guarani
        during and after the colonial period.
        \item \textbf{Post-Independence (1825):} Independence failed to
        improve conditions for indigenous people. The nation has struggled
        with internal cohesion, political instability (frequent coups), and
        regional demands for autonomy, especially from Santa Cruz.
    \end{itemize}

    \subsubsection{20th Century and Neoliberalism}

    \noindent \textbf{Tin Economy:} By the early 20th century, tin replaced
silver as the main export, making three `Tin Barons' exceptionally wealthy
and politically powerful.\\

    \noindent \textbf{1952 Nationalist Revolution:}
    \begin{itemize}
        \item A coalition of middle class, miners, and peasants overthrew
        the elite government.
        \item \textbf{Key Reforms:} Nationalized the mines, implemented land
        reform in the highlands, and granted universal suffrage to
        indigenous people and non-literate women.
        \item \textbf{State Policy:} The ruling \textbf{MNR} party sought to
        create a unified mestizo nation, renaming `Indians' as
        \textit{campesinos} (country people). The
        \textbf{COB} (Bolivian Worker's Central) labour confederation waas
        formed and became a powerful force for social justice.
    \end{itemize}

    \noindent \textbf{Military Dictatorships (19640-1982):}
    \begin{itemize}
        \item The army seized power, ushering in 18 years of repressive rule.
        \item The COB led the fight to restore democracy. The military
        maintained power partly through a pact with \textit{campesino}
        unions, promising to protect the 1952 land reforms.
    \end{itemize}

    \noindent \textbf{Neoliberal Era (1985-2005):}
    \begin{itemize}
        \item Democracy returned in 1982 amid economic collapse and
        hyperinflation (estimated at 25,000\%).
        \item In 1985, the government implemented a severe IMF-backed \textbf{New Economic Policy (NEP)}, a structural adjustment program that slashed public spending and liberalized trade.
        \item \textbf{Consequences:}
        \begin{itemize}
            \item The global tin price collapsed, leading to the closure of
            state mines and the firing of 23,000 miners, which
            \textbf{eviscerated the COB labour movement}.
            \item The workforce shifted from unionized formal labour to a
            largely informal one, with many displaced workers turning to
            coca cultivation in the Chapare region.
            \item \textbf{Decentralization (1994):} Channeled 20\% of the
            national budget to municipalities, expanding local political
            participation and strengthening indigenous identity.
        \end{itemize}
    \end{itemize}

    \noindent \textbf{Rise of Social Movements (2000s):}
    \begin{itemize}
        \item Neoliberalism's failure to reduce poverty led to widespread
        protest.
        \item \textbf{Cochabamba Water War (2000):} Successfully reversed the
        privatization of the public water company.
        \item \textbf{Gas Wars (2003, 2005):} Mass protests forced two
        presidents from office, demonstrating the power of social movements
        to challenge corporations and governments.
    \end{itemize}

    \subsubsection{The Country the MAS Inherited (2006):}
    \noindent \textbf{Urban Landscape:}
    \begin{itemize}
        \item Bolivia underwent a rapid transition from a rural to an urban
        society.
        \item \textbf{Santa Cruz:} Grew from 40,000 people in 1950 to 1.8
        million, becoming the country's largest city and economic hub for
        agribusiness and natural gas. It is marked by extreme inequality.
        \item \textbf{La Paz-El Alto:} A metropolis of 1.8 million.
        \begin{itemize}
            \item \textbf{El Alto} grew exponentially from displaced
            peasants and miners. By 2010, 74\% of its residents
            self-identified as Aymara. It is known as the hemisphere's
            ``most indigenous, most radical, and, ironicaly, most neoliberal
            city''.
        \end{itemize}
        \item \textbf{Cochabamba:} The country's former `granary', now a
        metropolitan area of 1 million, with an economy shaped by the nearby
        coca/cocaine trade.
    \end{itemize}

    \noindent \textbf{Challenges for the Morales Administration}
    \begin{itemize}
        \item Whe Evo Morales was elected, the traditional political parties
        were in disarray.
        \item However, his government faced five major challenges:
        \begin{itemize}
            \item [$1$.] Navigating powerful
            \textbf{transnational actors (World Bank, IMF, U.S.)}.
            \item[$2$.] Diffusing opposition from \textbf{domestic elites}
            and institutions (Senate, judiciary).
            \item[$3$.] Reforming a public service built on old
            \textbf{patronage systems}.
            \item[$4$.] Governing with a
            \textbf{small pool of trained professionals} while facing
            pressure to reward supporters with government posts.
            \item[$5$.] Managing \textbf{enormous expectations} for
            immediate and radical change from its poor and indigenous base.
        \end{itemize}
    \end{itemize}

    \subsection{Alina Duarte (2020) Bolivia and Necessary Self-Critique: ‘In it not enough to have the government, we have to have people’s power’ Council on Hemispheric Affairs. https://www.coha.org/bolivia-and-necessary-self-critique-it-is-not-enough-to-have-the-government-we-have-to-have-peoples-power}

    \subsubsection{Political Context: 2020 Election}

    \noindent \textbf{MAS Victory:} After a year-long de facto government
established by a coup, Luis Arce of the Movement Towards Socialism (MAS) won
the presidency with 55.11\% of the vote. The prior regime was characterized
by repression, racism, and corruption.\\

    \noindent \textbf{Morales's Return:} Former president Evo Morales
returned from exile in Argentina.\\

    \noindent \textbf{Strengthened Social Movements:} Resistance to the coup
invigorated and renewed social organizations.

\subsubsection{The Centrality of Self-Critique}

    \noindent \textbf{Core Lesson:} Self-criticism is presented as the most
crucial lesson and the strongest tool for MAS to resume and fortify the
``process of change''. It is necessary for understanding the errors that
allowed the coup to happen.\\

    \noindent \textbf{Return to Origins:} A primary challenge for MAS is to
return to its origins as a \textbf{`political instrument'} for popular
sovereignty, not just a conventional political party. The goal is to build \textbf{`communitarian socialism'} with the people.

    \subsubsection{Internal Analysis: Why the 2019 Coup Succeeded}

    \noindent \textbf{Over-reliance on Government:} A key mistake was the
belief that simply holding government power was enough.
    \begin{itemize}
        \item High economic growth (annual GDP growth of 4.9\% from
        2006-2019) and infrastructure projects were insufficient without an
        accompanying process of political formation and consciousness among
        the populace.
    \end{itemize}

    \noindent \textbf{Demobilization of Social Movements:}
    \begin{itemize}
        \item When an indigenous leader (Morales) took power, the `enemy' (
        capitalism, patriarchy, colonialism) was no longer a visible figure
        in office, leading to demobilization.
        \item Social movements became bureaucratized as they adopted the
        logic of wanting to be in office.
    \end{itemize}

    \noindent \textbf{Failure to Transcend Capitalism:} The government
focused on meeting basic needs within the capitalist market rather than
moving beyond it.\\

    \noindent \textbf{Internal Weaknesses:}
    \begin{itemize}
        \item The party's internal diversity, while a strength, also a
        ``generated a weakness because it has not strengthened the axis of
        discussion''.
        \item Some sectors of the administration shifted to the right,
        implementing contradictory policies.
    \end{itemize}




\end{document}