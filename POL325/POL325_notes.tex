\documentclass{article}

%settings
\usepackage[utf8]{inputenc}
\usepackage[explicit]{titlesec}
\usepackage{amsmath, amsfonts, amssymb, amsthm}
\usepackage{braket}
\usepackage[margin=1.0in]{geometry}
\usepackage{bbold}
\usepackage{fancyhdr}
\usepackage{fancyvrb}
\usepackage{graphicx}
\usepackage{float}
\usepackage{longtable}
\usepackage{array}
\usepackage{hyperref}
\pagestyle{fancy}
\fancyhead[L]{\leftmark}
\fancyhead[R]{\thepage}

\title{POL325: Contemporary Latin American Politics}
\author{Shamel Bhimani}
\date{Fall 2025}

\begin{document}

\maketitle

\tableofcontents
\newpage

\section{Development and its Alternatives}
    \subsection{Cristóbal Kay (2018) Modernization and Dependency Theory. From The Routledge Handbook of Latin American Development Julie Cupples, Marcel Palomino-Schalscha, and Manuel Prieto, eds. New York: Routledge, pp 15-28.}
    \subsubsection{Modernization Theory}
    \noindent \textbf{Historical Context:} Emerged in the North (
1950s-1960s) during the Cold War. Arose post-WWII as decolonization
accelerated.\\

    \noindent \textbf{Core Idea:} Development is a transition from a
`traditional' to a `modern' society. It posits that underdeveloped countries
can `catch up' by replicating the experience of Western, developed countries.
    \begin{itemize}
        \item Change is seen as determined by internal factors
    \end{itemize}

    \noindent \textbf{Dualistic Typology:}
    \begin{itemize}
        \item \textbf{Traditional Socieites:} Simple, rural subsistence
        economy, family labour, primitive technology, low productivity.
        Characterized by partciularism, adscription, and collective
        orientation. `Traditionalism' itself is seen as a barrier due to
        fatalistic outlook.
        \item \textbf{Modern Societies:} Complex, industrial,
        market-oriented, wage labour, scientific technology, high
        productivity. Characterized by universalism, achievement
        orientation, self-orientation, upward social mobility, and rule of law.
    \end{itemize}

    \noindent \textbf{Key Theorists and Concepts:}
    \begin{itemize}
        \item \textbf{Walt W. Rostow (1960):} Proposed five universal stages
        of economic growth:
        \begin{itemize}
            \item [$1$.] The traditional society;
            \item[$2$.] The preconditions for take-off;
            \item[$3$.] \textbf{Take-off (the key turning point)};
            \item[$4$.] The drive to maturity;
            \item[$5$.] The age of mass-consumption.
        \end{itemize}
        \item \textbf{Samuel Huntington (1968):} Priotized
        \textbf{political order and stability} above other modernization
        goals, concerned that rapid social change could overwhelm political
        institutions. Critiqued mainstream MT for being too static, arguing
        all societies combine traditional and modern elements.
        \item \textbf{Other theorisits focused on:} Value changes (Moore),
        personality transformation (Lerner), psychological factors like the
        desire to achieve (McClelland), and entreprenurial spirit (Hagen).
    \end{itemize}

    \noindent \textbf{Modernization Theory in Latin America:}
    \begin{itemize}
        \item Largely absorbed uncritically by Latin American social
        scientists and policymakers.
        \item \textbf{Gino Germani (1981):} A notable exception who adapted
        MT. He argued that transition processes create conflicts and
        `asynchronies' as different social spehres change at different
        speeds. However, his work was criticized from a Marxist perspective
        for failing to address class and ethnic conflicts.
    \end{itemize}

    \noindent \textbf{Critique of Modernization Theory (by Andre Gunder Frank):}
    \begin{itemize}
        \item Empirically faulty and theoretically weak.
        \item \textbf{Main flaw:} Assumes underdevelopment is an original
        state and ignores how development and underdevelopment are part of a
        single process in the formation of the world capitalist system since
        the 15th century.
        \item Fails to account for the impact of colonialism and imperialism.
    \end{itemize}

    \subsubsection{Dependence Theory (DT)}
    \noindent \textbf{Historical Context:} Arose in Latin America in the
mid-1960s, challenging MT. Influenced by theories of imperialism and the
Latin American structuralist school (ECLAC).\\

    \noindent \textbf{Core Idea:} Underdevelopment is not an original state
but a \textbf{`conditioning situation'} where the economies of some
countries (the periphery) are conditioned by the development and expansion
of others (the center).
    \begin{itemize}
        \item Development and underdevelopment are seen as two faces of the
same historical process of global capitalism.
        \item It analyzes the link between external (global capitalism) and
        internal (class structure, politics) factors.
    \end{itemize}

    \noindent \textbf{Precursor: Raul Prebisch and ECLAC Structuralism:}
    \begin{itemize}
        \item Developed the \textbf{center-periphery paradigm}.
        \item \textbf{Prebisch-Singer Thesis:} Argued the international
        trade system benefits the center at the expense of the periphery due
        to the long-term deterioration of the periphery's terms of trade (
        prices of its primary commodity exports fall relative to the
        industrial goods it imports).
    \end{itemize}

    \noindent \textbf{Main Strands of Dependence Theory:}
    \begin{itemize}
        \item [$1$.] \textbf{Structuralis Strand:} Seeks to reform the
        capitalist system. Uses heterodox development theory concepts.
        \item[$2$.] \textbf{Marxist Strand:} Argues dependency can only be
        overcome by overthrowing capitalism adn transitioning to socialism.
        Relies on historical materialism and the labour theory of value.
    \end{itemize}

    \noindent \textbf{Key Structuralist Theorists and Concepts:}
    \begin{itemize}
        \item \textbf{Osvaldo Sunkel:} Focused on how transnational
        corporations (TNCs) deepen dependece and cause
        \textbf{`national disintegration'}. TNCs weaken the national
        bourgeoisie, fragment society, and shape public policy against the
        national interst.
        \item \textbf{Celso Furtado:} Analyzed
        \textbf{`dependent consumption patterns'}. The consumption habits of
        the rich, influenced by developed countries, create a wasteful,
        capital-intensive, and import-demanding industrial structure that
        perpetuates income concentration and underdevelopment.
        \item \textbf{Fernando Henrique Cardoso and Enzo Faletto:}
        \begin{itemize}
            \item Emphasized analyzing \textbf{`situations of dependency'}
            rather than a single theory, focusing on internal manifestations.
            \item Characterized the process as
            \textbf{`dependent development'}, rejecting stagnationist views
            and acknowledging that economic growth could occur within a
            dependency framework.
            \item Argued a \textbf{`new dependency'} emerged under
            corporatist-authoritarian states controlled by a militarized
            technocratic bureaucracy.
        \end{itemize}
    \end{itemize}

    \noindent \textbf{Key Marxist Theories and Concepts:}
    \begin{itemize}
        \item \textbf{Theotonio Dos Santos:} Identified a
        \textbf{`new character of dependency'} rooted in industrial and
        technological dependence. The lack of a domestic capital goods
        industry and indigenous technological capacity prevents dependent
        economies form being `articulated' and achieving autonomous development.
        \item \textbf{Ruy Mauro Marini:} Focused on \textbf{unequal exchange} (transfer of surplus value to dominate countries) and the resulting \textbf{over-exploitation of labour} in dependent countries to maintain profit rates. Also developed the concept of \textbf{sub-imperialism}, where a larger dependent country like Brazil under its military regime could engage in imperialist practices toward weaker neighbours to solve its own problems of insufficient internal demand.
        \item \textbf{Andre Gunder Frank:} Coined the phrase
        \textbf{``the development of underdevelopment}. Argued that the
        metropolis-satellite linkages constantly recreate underdevelopment.
        Contended that Latin America has been capitalist since the colonial
        conquest, challenging the prevailing feudal/semi-feudal
        characterization and the political strategies based on it. His
        thesis was famously critiqued by Ernesto Laclau for overemphasizing
        market circulation while neglecting relations of production.
    \end{itemize}

    \noindent \textbf{Decline and Legacy of Dependence Theory:} It's
influence waned with the economic crisis of the 1980s and the rise of
neoliberalism. Remebered as the
\textbf{first major challenge to the Eurocentric character of the social sciences} to achieve global influence. It inspired a new generation of scholars to think about development from the perspective of the South.

\subsection{Laura Zapata-Cantu and Fernando González (2021) Challenges for Innovation and
Sustainable Development in Latin America: The Significance of Institutions and
Human Capital. Sustainability.}

    \subsubsection{Innovation and Sustainable Development in Latin America: Core Challenges}

    \noindent \textbf{Context:} Sustainable development is a critical 21st
century challenge, compounded by the COVID-19 pandemic which has shifted
government and business priorities. Innovation is essential for
transitioning to a more sustainable world.\\

    \noindent \textbf{Regional Challenges:}
    \begin{itemize}
        \item \textbf{High social inequality and poverty} remain significant
        obstacles.
        \item The region has a history of
        \textbf{political instability and corruption}, which impacts
        institutional performance in innovation and sustainability.
        \item Economies are vulnerable to \textbf{commodity price volatility} and uncertainty.
        \item There is a \textbf{marginal contribution to global innovation}, often measured by patent registrations.
        \item Latin American countries face ongoing issues with nutrition,
        sanitation, quality education, and economic modernization.
    \end{itemize}

    \subsubsection{Theoretical Framework for Analysis}

    \noindent Two key theoretical perspectives are used to analyze
innovation and sustainable development in the region:
    \begin{itemize}
        \item [$1$.] \textbf{Dynamic Capabilities (DCV):} A framework
        focusing on a
        country's or firm's ability to adapt and transform in response to a
        changing environemnt. It is composed of three core capabilities:
        \begin{itemize}
            \item \textbf{Sensing:} Identifying opportunities and threats by
            scanning markets and technologies. THis aligns with the Global
            Innovation Index (GII) dimensions of \textit{Institutions} and \textit{Market Sophistication}.
            \item \textbf{Seizing:} Mobilizing resources to capture
            opportunities by investing in technology, human capital, and new
            business models. This aligns with the GII dimensions of
            \textit{Human capital and research, Infrastructure, and Business sophistication}.
            \item \textbf{Transforming:} Continuously reconfiguring assets
            and organizational structures to maintain the competitiveness.
            This aligns with the GII dimensions of
            \textit{Knowledge and technology outputs and creative outputs}.
        \end{itemize}
        \item[$2$.] \textbf{Mission-Oriented Policies:} Systemic public
        sector initiatives aimed at solving specific societal problems (
        e.g., climate change) by mobilizing innovation across multiple
        sectors and actors.
        \begin{itemize}
            \item These policies define a clear direction, foster
            collaboration between public and private sectors, and drive
            technological and systemic change.
            \item They require strong institutional support and the
            development of dynamic capabilities within government to lead
            change.
        \end{itemize}
    \end{itemize}

\end{document}