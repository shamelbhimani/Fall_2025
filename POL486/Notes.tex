\documentclass{article}

%settings
\usepackage[utf8]{inputenc}
\usepackage[explicit]{titlesec}
\usepackage{amsmath, amsfonts, amssymb, amsthm}
\usepackage{braket}
\usepackage[margin=1.0in]{geometry}
\usepackage{bbold}
\usepackage{fancyhdr}
\usepackage{fancyvrb}
\usepackage{graphicx}
\usepackage{float}
\usepackage{longtable}
\usepackage{array}
\usepackage{hyperref}
\pagestyle{fancy}
\fancyhead[L]{\leftmark}
\fancyhead[R]{\thepage}

\title{POL486: Networks in International Politics}
\author{Shamel Bhimani}
\date{Fall 2025}

\begin{document}

\maketitle

\tableofcontents
\newpage

    \section{Introduction to Network Analysis}
    \subsection{Olga V. Chyzh. Network analysis in international relations. In Cameron G. Thies, editor, Handbook of International Relations, pages 158–170. Edward Elgar Publishing, 2025.}
    \subsubsection{Networks in International Relations}
    \noindent \textbf{Core Premise:} International politics is inherently
networked. Actors (states, organizations, individuals) are interconnected
nodes, and their relationships (alliances, trade, conflict) are ties.\\

    \noindent \textbf{Significance of Networks:}
    \begin{itemize}
        \item Membership in international clubs (e.g., NATO, EU, WTO) offers
        security, prestige, and economic benefits.
        \item Exclusion can lead to insecurity and foreign policy revisionism.
        \item Network embeddedness affects policy options and resource access.
    \end{itemize}

    \noindent \textbf{Historical Context:} IR scholars have long recognized
the networked nature of global politics, but network analysis provided the
specific tools to better align theory with empirical evidence.\\

    \noindent \textbf{Critique of Traditional IR Research:}
    \begin{itemize}
        \item Network scholars criticized previous IR research, particularly
        the \textbf{dyadic design}, for its failure to account for
        interdependence among actors.
        \item The assumption of independence in dyadic analysis can lead to
        confounding bias, attributing effects to incorrect causes. For
        example, the US-Japan and US-South Korea relationships influence the
        Japan-South Korea relationship.
    \end{itemize}

    \subsubsection{Methodological and Theoretical Contributions}
    \noindent \textbf{Primary Contribution:} Network analysis offered a way
to measure previously unmeasurable concepts like system polarity, social
power, and prestige. It also generated new research questions about
connectivity.\\

    \noindent \textbf{Three Main Research Approaches:}
    \begin{itemize}
        \item \textbf{Global Network Properties:} Studies focus on
        properties of the entire network, such as density of
        fractionalization, to explain outcomes like conflict and cooperation.
        \begin{itemize}
            \item [$1$.]\textbf{Example:} Maoz (2006) used network analysis to
            create a theory-informed measure of international system
            polarization, a concept previously hard to operationalize.
            \item \textbf{Example:} Cruz, Labonne, and Querubin (2020) found
            that greater fractionalization (power divided among more clans)
            in local kinship networks in the Philippines was associated with
            better public goods provision.
        \end{itemize}
        \item [$2$.] \textbf{Actor Positions:} This approach analyzes the
        positions of individual actors within a network to understand power
        and influence.
        \begin{itemize}
            \item \textbf{Brokerage/Gate-keeping Power:} Held by actors
            connecting otherwise discontinued clusters.
            \item \textbf{Network Centrality:} Used as a proxy for concepts
            like country prestige (Renshon 2016) or an organization's
            agenda-setting power (Carpenter 2011).
        \end{itemize}
        \item[$3$.] \textbf{Overlapping Membership:} This apporach examines
        how overlapping memberships reinforce each other's effects.
        \begin{itemize}
            \item \textbf{Example:} Parkisnon (2013) showed that sustaining
            an insurgency depends on the overlap between military networks
            and personal networks (kinship, friendship).
            \item \textbf{Example:} Eldredge and Shannon (2022) found that
            countries with high membership overlap in inter-governmental
            organizations are more likely to object to each other's human
            rights treaty reservations.
        \end{itemize}
    \end{itemize}

    \subsubsection{The Debate and Normalization of Network Analysis in IR}
    \noindent \textbf{The `Us-vs-Them' Debate:} Early proponents of network
analysis adopted a provocative framing, creating divisions within IR.\\

    \noindent \textbf{Points of Resistance:} Critics argued that the
traditional dyadic approach had not impeded major theoretical advances (
e.g., the democratic peace) and that research designs should be tailored to
the specific question, rather than assuming interdependence as the default.\\

    \noindent \textbf{Mainstreaming the Approach (c. 2016):}
    \begin{itemize}
        \item An exchange in \textit{International Relations Quarterly}
        between proponents and critics marked a key moment.
        \item A special issue on networks in
        \textit{Journal of Peace Research} showcased the breadth of
        applications.
        \item The field of political methodology quickly welcomed and
        published inferential network analysis research.
        \item Subsequently, network research began appearing in top
        disciplinary journals, at major conferences, and in university
        curricula.
    \end{itemize}

    \subsubsection{Current Research and Future Directions}
    \noindent \textbf{Addressing Endogeneity:} Developing tools to separate
actor-level effects from network-level effects (e.g., democracy vs. clique
size in trade).\\

    \noindent \textbf{Flexible Conceptualization:} Re-evaluating the unit of
analysis,
such as treating alliances themselves as nodes to study action-reaction
processes.\\

    \noindent \textbf{Expanding Scope:} Applying network analysis to
subnational and transnational levels, including rebel groups, NGOs, and
political elites.\\

    \noindent \textbf{Social Media Data:} Utilizing vast, inherently
networked data from social media to study mobilization, censorship, and
misinformation.\\

    \noindent \textbf{Future Directions:}
    \begin{itemize}
        \item \textbf{Develop IR-Specific Theories:} Move beyond borrowing
        sociological theories to build network theories tailored to IR's
        unique actors and assumptions (e.g., anthropomorphizing states).
        \item \textbf{Model Hierarchical Networks:} Incorporate asymmetrical
        and hierarchical relationships, not just horizontal ones between
        equal actors.
        \item \textbf{Integrate Casual Inference:} Bridge network analysis
        with experimental and quasi-experimental methods to test network
        predictions more rigorously.
        \item \textbf{Model Co-evolution:} Better theorize and moel the
        endogenous relationship where actor characteristics are both a cause
        and an effect of their network ties.
    \end{itemize}


    \section{Network Centrality}
    \subsection{John F. Padgett and Christopher K. Ansell. Robust action and the rise of the Medici, 1400-1434. American Journal of Sociology, 98(6):1259–1319, 1993.}
    \subsubsection{Introduction and Core Argument}

    \noindent \textbf{Central Claim:} The rise of Medicean political
control (1400-1434) in Florence, leading to the Renaissance state, was
driven by \textbf{network disjunctures} within the elite that the Medici
alone spanned.\\

    \noindent \textbf{Methodology:} To understand state formation, one must
move beyond formal institutions, groups, and goals to the
\textbf{relational substrata} of people's actual lives.
\textbf{Ambiguity and heterogeneity}, not planning and self-interest, are
the raw materials of powerful states and persons.\\

    \noindent \textbf{Key Concept:} \textit{Robust Action}, Cosimo de
Medici's (1380-1464) control style, characterized by
\textbf{multivocal identity as `sphinx'}, which harnessed power from these
`network holes' and resolved the inherent contradiction between `judge' and
`boss' in organizations.

    \subsubsection{Core Concepts Explained}

    \noindent \textbf{Political Centralization and the Judge/Boss Contradiction}
    \begin{itemize}
        \item State-building involves centralizing power, a contradictory
        process requiring both reproduction (rules creating roles,
        interests, collective action patterns) and control (others'
        interactions serving one's interests).
        \item The contradiction: A founder cannot be both an impartial
        `judge' (legitimacy through non-self-interest) and a controlling
        `boss' (direct intervention undermines legitimacy).
    \end{itemize}

    \noindent \textbf{Robust Action:}
    \begin{itemize}
        \item \textbf{Definition:} A style of control where single actions
        can be coherently interpreted from multiple perspectives
        simultaneously (multivocality), serve as moves in many `games' at
        once, and blur public/private motivations.
        \item \textbf{Mechanism:} Leads to `Rorschach blot identities',
        where others attribute their own distrinctive identity to the ego (
        Cosimo).
        \item \textbf{Goal:} Maintaining \textbf{flexible opportunism} and
        discretionary options in unpredictable futures, rather than pursuing
        specific, fixed goals. This involves \textbf{positional play:}
        maneuvering opponents into clarifying \textit{their} (not your)
        tactical lines of action.
        \item \textbf{Resolution of Judge/Boss:} Credible robust action
        works because the center (Cosimo) appears to have
        \textbf{no unequivocal self-interests}; he `merely' responds to
        requests. Control is diffused, as others infer and serve his
        inscrutable interests.
        \item \textbf{Preconditions:} Requires specific network structures
        for channeling requests and opaque, coherent interests to be credible.
    \end{itemize}


\end{document}