\documentclass{article}

%settings
\usepackage[utf8]{inputenc}
\usepackage[explicit]{titlesec}
\usepackage{amsmath, amsfonts, amssymb, amsthm}
\usepackage{braket}
\usepackage[margin=1.0in]{geometry}
\usepackage{bbold}
\usepackage{fancyhdr}
\usepackage{fancyvrb}
\usepackage{graphicx}
\usepackage{float}
\usepackage{longtable}
\usepackage{array}
\usepackage{hyperref}
\pagestyle{fancy}
\fancyhead[L]{\leftmark}
\fancyhead[R]{\thepage}

\title{POL486: Networks in International Politics}
\author{Shamel Bhimani}
\date{Fall 2025}

\begin{document}

\maketitle

\tableofcontents
\newpage

    \section{Introduction to Network Analysis}
    \subsection{Olga V. Chyzh. Network analysis in international relations. In Cameron G. Thies, editor, Handbook of International Relations, pages 158–170. Edward Elgar Publishing, 2025.}
    \subsubsection{Networks in International Relations}
    \noindent \textbf{Core Premise:} International politics is inherently
networked. Actors (states, organizations, individuals) are interconnected
nodes, and their relationships (alliances, trade, conflict) are ties.\\

    \noindent \textbf{Significance of Networks:}
    \begin{itemize}
        \item Membership in international clubs (e.g., NATO, EU, WTO) offers
        security, prestige, and economic benefits.
        \item Exclusion can lead to insecurity and foreign policy revisionism.
        \item Network embeddedness affects policy options and resource access.
    \end{itemize}

    \noindent \textbf{Historical Context:} IR scholars have long recognized
the networked nature of global politics, but network analysis provided the
specific tools to better align theory with empirical evidence.\\

    \noindent \textbf{Critique of Traditional IR Research:}
    \begin{itemize}
        \item Network scholars criticized previous IR research, particularly
        the \textbf{dyadic design}, for its failure to account for
        interdependence among actors.
        \item The assumption of independence in dyadic analysis can lead to
        confounding bias, attributing effects to incorrect causes. For
        example, the US-Japan and US-South Korea relationships influence the
        Japan-South Korea relationship.
    \end{itemize}

    \subsubsection{Methodological and Theoretical Contributions}
    \noindent \textbf{Primary Contribution:} Network analysis offered a way
to measure previously unmeasurable concepts like system polarity, social
power, and prestige. It also generated new research questions about
connectivity.\\

    \noindent \textbf{Three Main Research Approaches:}
    \begin{itemize}
        \item \textbf{Global Network Properties:} Studies focus on
        properties of the entire network, such as density of
        fractionalization, to explain outcomes like conflict and cooperation.
        \begin{itemize}
            \item [$1$.]\textbf{Example:} Maoz (2006) used network analysis to
            create a theory-informed measure of international system
            polarization, a concept previously hard to operationalize.
            \item \textbf{Example:} Cruz, Labonne, and Querubin (2020) found
            that greater fractionalization (power divided among more clans)
            in local kinship networks in the Philippines was associated with
            better public goods provision.
        \end{itemize}
        \item [$2$.] \textbf{Actor Positions:} This approach analyzes the
        positions of individual actors within a network to understand power
        and influence.
        \begin{itemize}
            \item \textbf{Brokerage/Gate-keeping Power:} Held by actors
            connecting otherwise discontinued clusters.
            \item \textbf{Network Centrality:} Used as a proxy for concepts
            like country prestige (Renshon 2016) or an organization's
            agenda-setting power (Carpenter 2011).
        \end{itemize}
        \item[$3$.] \textbf{Overlapping Membership:} This apporach examines
        how overlapping memberships reinforce each other's effects.
        \begin{itemize}
            \item \textbf{Example:} Parkisnon (2013) showed that sustaining
            an insurgency depends on the overlap between military networks
            and personal networks (kinship, friendship).
            \item \textbf{Example:} Eldredge and Shannon (2022) found that
            countries with high membership overlap in inter-governmental
            organizations are more likely to object to each other's human
            rights treaty reservations.
        \end{itemize}
    \end{itemize}

    \subsubsection{The Debate and Normalization of Network Analysis in IR}
    \noindent \textbf{The `Us-vs-Them' Debate:} Early proponents of network
analysis adopted a provocative framing, creating divisions within IR.\\

    \noindent \textbf{Points of Resistance:} Critics argued that the
traditional dyadic approach had not impeded major theoretical advances (
e.g., the democratic peace) and that research designs should be tailored to
the specific question, rather than assuming interdependence as the default.\\

    \noindent \textbf{Mainstreaming the Approach (c. 2016):}
    \begin{itemize}
        \item An exchange in \textit{International Relations Quarterly}
        between proponents and critics marked a key moment.
        \item A special issue on networks in
        \textit{Journal of Peace Research} showcased the breadth of
        applications.
        \item The field of political methodology quickly welcomed and
        published inferential network analysis research.
        \item Subsequently, network research began appearing in top
        disciplinary journals, at major conferences, and in university
        curricula.
    \end{itemize}

    \subsubsection{Current Research and Future Directions}
    \noindent \textbf{Addressing Endogeneity:} Developing tools to separate
actor-level effects from network-level effects (e.g., democracy vs. clique
size in trade).\\

    \noindent \textbf{Flexible Conceptualization:} Re-evaluating the unit of
analysis,
such as treating alliances themselves as nodes to study action-reaction
processes.\\

    \noindent \textbf{Expanding Scope:} Applying network analysis to
subnational and transnational levels, including rebel groups, NGOs, and
political elites.\\

    \noindent \textbf{Social Media Data:} Utilizing vast, inherently
networked data from social media to study mobilization, censorship, and
misinformation.\\

    \noindent \textbf{Future Directions:}
    \begin{itemize}
        \item \textbf{Develop IR-Specific Theories:} Move beyond borrowing
        sociological theories to build network theories tailored to IR's
        unique actors and assumptions (e.g., anthropomorphizing states).
        \item \textbf{Model Hierarchical Networks:} Incorporate asymmetrical
        and hierarchical relationships, not just horizontal ones between
        equal actors.
        \item \textbf{Integrate Casual Inference:} Bridge network analysis
        with experimental and quasi-experimental methods to test network
        predictions more rigorously.
        \item \textbf{Model Co-evolution:} Better theorize and moel the
        endogenous relationship where actor characteristics are both a cause
        and an effect of their network ties.
    \end{itemize}


    \section{Network Centrality}
    \subsection{John F. Padgett and Christopher K. Ansell. Robust action and the rise of the Medici, 1400-1434. American Journal of Sociology, 98(6):1259–1319, 1993.}
    \subsubsection{Introduction and Core Argument}

    \noindent \textbf{Central Claim:} The rise of Medicean political
control (1400-1434) in Florence, leading to the Renaissance state, was
driven by \textbf{network disjunctures} within the elite that the Medici
alone spanned.\\

    \noindent \textbf{Methodology:} To understand state formation, one must
move beyond formal institutions, groups, and goals to the
\textbf{relational substrata} of people's actual lives.
\textbf{Ambiguity and heterogeneity}, not planning and self-interest, are
the raw materials of powerful states and persons.\\

    \noindent \textbf{Key Concept:} \textit{Robust Action}, Cosimo de
Medici's (1380-1464) control style, characterized by
\textbf{multivocal identity as `sphinx'}, which harnessed power from these
`network holes' and resolved the inherent contradiction between `judge' and
`boss' in organizations.

    \subsubsection{Core Concepts Explained}

    \noindent \textbf{Political Centralization and the Judge/Boss Contradiction}
    \begin{itemize}
        \item State-building involves centralizing power, a contradictory
        process requiring both reproduction (rules creating roles,
        interests, collective action patterns) and control (others'
        interactions serving one's interests).
        \item The contradiction: A founder cannot be both an impartial
        `judge' (legitimacy through non-self-interest) and a controlling
        `boss' (direct intervention undermines legitimacy).
    \end{itemize}

    \noindent \textbf{Robust Action:}
    \begin{itemize}
        \item \textbf{Definition:} A style of control where single actions
        can be coherently interpreted from multiple perspectives
        simultaneously (multivocality), serve as moves in many `games' at
        once, and blur public/private motivations.
        \item \textbf{Mechanism:} Leads to `Rorschach blot identities',
        where others attribute their own distrinctive identity to the ego (
        Cosimo).
        \item \textbf{Goal:} Maintaining \textbf{flexible opportunism} and
        discretionary options in unpredictable futures, rather than pursuing
        specific, fixed goals. This involves \textbf{positional play:}
        maneuvering opponents into clarifying \textit{their} (not your)
        tactical lines of action.
        \item \textbf{Resolution of Judge/Boss:} Credible robust action
        works because the center (Cosimo) appears to have
        \textbf{no unequivocal self-interests}; he `merely' responds to
        requests. Control is diffused, as others infer and serve his
        inscrutable interests.
        \item \textbf{Preconditions:} Requires specific network structures
        for channeling requests and opaque, coherent interests to be credible.
    \end{itemize}

    \subsubsection{Historical Context: Florence (1400-1434)}

    \noindent \textbf{Transition:} From late medieval urban factionalism
to a regionally consolidated Renaissance state.\\

    \noindent \textbf{Ultimate Causes:} Unsuccessful class revolt (Ciompi
revolt, 1378-82) and severe fiscal crisis due to wars (Milan and Lucca wars,
1424-33).\\

    \noindent \textbf{Medici's Rise:} Cosimo de Medici (1389-1464) founded a
dynasty, consolidated a Europe-wide banking network, and sponsored the
Renaissance.\\

    \noindent \textbf{Cosmio's `Sphinx-Like' Character:}\
    \begin{itemize}
        \item Contemporaries deeply appreciated his power, yet eyewitness
        accounts describe him as indecipherable.
        \item He remained in the background, acting through deputies, with
        little known of his direct responsibilities.
        \item He never assumed lasting public office and rarely gave public
        speeches.
        \item His actions appeared reactive, serving his `extremely multiple
        interests'.
        \item His replies were often brief and obscure, sometimes Delphic or
        using proverbs, allowing double interpretations.
    \end{itemize}

    \subsubsection{Methodology and Data}

    \noindent \textbf{Approach:} An `archaeological dig' into the structural
preconditions for Medici's success, focusing on the composition and social
network structure of the Medici party versus their opponents, the `oligarchs'.\\

    \noindent \textbf{Sources:} Based on extensive historical work,
especially Dale Kent's `The Rise of the Medici' (1978). Supplemented by
1427/1433 catasto (tax registers), Najemy (1982), and Martines (1963).\\

    \noindent \textbf{Data Types (9 types of relations among elite families):}
    \begin{itemize}
        \item \textbf{Kinship:} Intermarriage ties (assymetric, 1394-1434).
        \item \textbf{Economic:} Trading/business ties, joint ownerships,
        bank employment, real estate ties.
        \item \textbf{Political:} Patronage and personal loans (multifaceted
        motives).
        \item \textbf{Personal:} Friendship, mallevadori (surety) ties.
    \end{itemize}

    \noindent \textbf{Definitions:}
    \begin{itemize}
        \item \textbf{Family:} Operationalized as `people with a common last
        name' (clan), consistent with Florentine social reality and data
        limitations.
        \item \textbf{Elite:} Families meeting any of three criteria:
        \begin{itemize}
            \item 2+ members speaking in Consulte e Prtiche (1429-34);
            \item 3+ members qualified for scrutiny (election to high
            office) in 1433;
            \item Magnate clan (215 families identified, 92 for network
            analysis).
        \end{itemize}
        \item \textbf{Blockmodel Analysis:} Aggregates actors (families)
        into structurally equivalent `blocks' based on common external ties
        with outsiders, not dense internal relations (cliques). Used to
        visualize marriage and economic networks (`strong ties', fig. 2a)
        and political/friendship networks (`weak ties', fig 2b).
    \end{itemize}

    \subsubsection{Empirical Findings: Attributional vs. Network Structure}

    \noindent \textbf{Attributional Analyses (Challenging Traditional Views):}
    \begin{itemize}
        \item \textbf{Economic Class (Wealth/Change in Wealth):} Both
        Mediceans and oligarchs were wealthy, but their wealth distributions
        were statistically identical and highly heterogenous. Not a Marxist
        class struggle.
        \item \textbf{Social Class (Presitge/Political Age):} Oligarch were
        more skewed towards older participants due to the \textit{absence}
        of `new men' from their party, not absence of particians from the
        Medicean side. Mediceans were more socially heterogenous, and
        relative to neutrals, were distinctly `old-guard patrician'.
        \item \textbf{Neighbourhood Residence:} No statistically significant
        difference, both parties mirrored each other in geographical
        concentration, especially in San Giovanni. San Giovanni was the most
        polarized quarter.
        \item \textbf{Conclusion:} There was a \textbf{structural mismatch}
        between contemporaries' clear cognitives typifications (oligarchs as
        old, wealth patricians; Mediceans as heroes of rising new men) and
        the objective heterogeneity and overlap of social groups at the
        behavioural level. Classical group theories of parties (
        pluralist/neo-Marxist) are insufficient.
    \end{itemize}

    \noindent \textbf{Social Network Structure (Blockmodel Analysis):}
    \begin{itemize}
        \item \textbf{Predictive Power:} Marriage and economic blockmodels
        remarkably predict political partisanship, despite attributional
        identity. The Medici family itself bridged both sides.
        \begin{itemize}
            \item 93\% of families within the `Medicean circle' were Medici
            partisans.
            \item 82\% of other partisan families (excluding neutrals)
            joined the oligarch side.
        \end{itemize}
        \item \textbf{Medici Party Structure:} An extraordinarily
        \textbf{centralized `star' or `spoke' network system}.
        \begin{itemize}
            \item Medici partisans were connected to other partisans and the
            oligarch elite almost solely \textit{through the Medici}.
            \item Medici partisans had remarkably few intraelite network
            ties, being `structurally impoverished'.
        \end{itemize}
        \item \textbf{Oligarch Party Structure:} Densely interconnected,
        especially through marriage, but this density led to
        \textbf{cacophony and cross-pressure}, not cohesive collective
        action (e.g., Rinaldo Albizzi's failed mobilization).
        \item \textbf{The Structural Atomization Puzzle:} Why did this
        centralized spoke system maintain itself.
        \begin{itemize}
            \item Medici followers had clear incentives to form cross-ties
            to alleviate dependence.
            \item The Medici discouraged
            \textbf{multiplex ties} (overlapping marriage and economic relations) with their followers, and also segregation of types of ties with the Medici themselves.
        \end{itemize}
        \item \textbf{Resolution: Double Segregation of Attributes}
        \begin{itemize}
            \item \textbf{Patrician Supporters:} Wealthy, old patricians (
            e.g., Guicciardini, Tornabuoni blocks) intermarried with the
            Medici, but resided \textit{outside} the San Giovanni quarter.
            \item \textbf{New Men Supporters:} Connected to Medici through
            economic or personal loan ties (e.g., Ginori, Orlandini,
            Cocco-Donati blocks), but residede \textit{within} San Giovanni.
            \item \textbf{Mechanism:} Patricians and new men despised each
            other and had limited interaction. Only the Medici linked these
            segments. This structural isolations inhibited defensive
            counter-alliances (`revolt of the colonels').
            \item \textbf{Medici Strategy:}
            \begin{itemize}
                \item In marriage and friendship, Medici were highly
                selective (snobbish, marrying other patricians).
                \item In the economic sphere, they associated heavily with
                new men, unlike other elite families.
                \item Their distinctiveness was
                \textit{associating with enw men at all}, not representing them.
            \end{itemize}
            \item \textbf{New Men's Responsiveness:} 90-96\% of new men
            economically/politically tied to Medici became active partisans.
            This was not due to active Medici mobilization of new men as a
            whole, but the \textit{oligarchs extraordinary inaction} towards
            them, leaving new men `structurally available'. Oligarchs'
            polemics branded Medici as `class traitors' (``heroes of then
            new men'').
        \end{itemize}
        \item \textbf{Control Mechanisms:}
        \begin{itemize}
            \item Spoke structure ensured dependence and channeled
            communication through Medici.
            \item Double segregation prevented counter-alliances among
            partisans.
            \item Formal affine relations with distant patricians (less
            frequent contact) contrasted with friendly, useful business ties
            with local new men (where status gap ensured deference).
            \item Attributional heterogeneity made Medici a potent `swing vote'.
            \item \textbf{Contradiction was key to control}, especially with
            intense surrounding cognitive group identities.
        \end{itemize}
    \end{itemize}

    \subsubsection{Network Dynamics: How the Medici Party Emerged}

    \noindent \textbf{No Grand Design:} Cosimo did not design his party or
initially intend to take over the state. The network patterns emerged from
oligarchs' previous actions and inadvertently channeled material to the Medici.

    \begin{itemize}
        \item [$1$.] \textbf{Dynamics of Patrician Marriage (1385-1420):}
        \begin{itemize}
            \item \textbf{Context:} Oligarchs' reconsolidation of control
            after the Ciompi revolt (1378).
            \item \textbf{Historical Trend:} Increasing rates of
            \textbf{neighbourhood exogamy} in Florentine elite marriage.
            This dissolved older `neighbourhood solidarity' mode of elite
            organization (quasi-feudal, intra-neighbourhood hierarchies).
            \item \textbf{Elite Closure (Post-Ciompi, 1382):}
            \begin{itemize}
                \item \textbf{Shunning `Class Traitors':} Patrician
                families (like Medici) who sympathized with Ciompi were
                severely ostracized in marriage by victorious oligarchs.
                This created the \textbf{structural barrier} seen in Figure
                2a between oligarchs and Medicean patrician blocks (
                Guicciardini, Tornabuoni).
                \item
                \textbf{Oligarch Co-optation and Cross-Neighbourhood Cycles:} Oligarchs began forming \textbf{cross-neighbourhood marraige cycles} to co-opt potentially bridging ``swing vote'' families (e.g., Rondinelli). This created a dense, citywide elite, closing in on itself.
                \item \textbf{Outcast Patrician Exogamy:} Structurally
                isolated, outcast patricians (like Medici) were forced to
                marry fellow isolates \textit{outside} their neighbourhoods
                to preserve status, leading to higher exogamy rates among them.
            \end{itemize}
            \item \textbf{Medici's Anomalous Position:}
            \begin{itemize}
                \item \textbf{Survival:} Veieri di Cambio's (Medici clan
                head) defusion of a pro-Alberti revolt in 1393 and Giovanni
                de Medici's later circumspection (avoiding politics,
                squelching discontent) saved the Medici name from utter
                ostracism. This earned them begrudging oligarch gratitude.
                \item \textbf{Limited Co-optation:} Oligarchs slowly
                relented in the 1420s, allowing some intermarriage with
                Medici (e.g., Albizzi, Gianfigliazzi blocks), but only after
                the Medici were already deeply isolated.
                \item \textbf{Exploiting Structural Holes:} Oligarchs
                over-focused on containing San Giovanni (Ricci's old home),
                creating a `structural hole' in the Santo Spirito quarter.
                The Medici gradually exploited this, directing 100\% of
                their own marriages to Santo Spirito by the early 1430s,
                often `wife-receiving' (less status-picky).
            \end{itemize}
            \item \textbf{Adaptive Learning:} Elite tactics evolved not from
            grand strategies but as a mutually adaptive learning process,
            with families making `boundedly rational local action' from
            their egocentric network positions.
        \end{itemize}
        \item [$2$.] \textbf{Dynamics of New Men Economic Ties (1420s-1430s):}
        \begin{itemize}
            \item \textbf{Catalyst:} Milan and Lucca wars (1424-33) led to
            devastating tax extraction, threatening family patrimonies.
            \item \textbf{Neighbourhood Politics Revival:} Tax assessments
            by neighbourhood intensified local politics.
            \item \textbf{Oligarch Repression:} Patricians legislatively
            targeted new men; successful repression abolished new men's
            nascent corporate forms (religious confraternities), leaving
            them without local support.
            \item \textbf{Medici as Exception:} Locked in by their dense
            marriage network, most oligarchs rejected appeals from new men.
            The Medici, with their `structurally contradictory position',
            had the discretion to respond to pleas from San Giovanni new men.
            \item \textbf{Medici Self-Consciousness:} The surge of
            supplication from San Giovanni new men during the Milan war
            galvanized the Medici into self-awareness as a political party.
            Oligarch actions (e.g., Rinaldo Albizzi's class alliance request
            to Giovanni de Medici, which Giovanni equivocally refused)
            further solidified the Medici as a distinct faction.
        \end{itemize}
    \end{itemize}

    \subsubsection{Network Identities: Robust Action and Legitimacy}

    \noindent \textbf{Credibility of Robust Action:} The contradictory
attributions of `Medici self-interest' (hero of new men vs. patrician) were
credible because the disparate groups of Medici supporters rarely hd the
opportunity to compare notes privately. Even if they had, low trust would
have prevented agreement.\\

    \noindent \textbf{Opaque Self-Interests:} Medici goals (money, prestige,
power) were tied to specific roles, not an overarching utility function. In
chaotic times, the ``games themselves are all up for grabs'', making
revealed preferences impossible to infer \textit{a priori}. Cosimo's and
Giovanni's ``shrewed and multivocal opportunism'' was a feature of varying
game structures, not fixed personal goals.\\

    \noindent \textbf{Legitimacy (Pater Patriae):}
    \begin{itemize}
        \item Cosimo was enshrined as \textit{pater patriae} upon his death,
        transmuting his ambiguity into public interest.
        \item He achieved this not by directly defeating oligarchs or
        slinging mud, but by \textbf{positional maneuvering}.
        \item His `reactive character' forced oligarchs into aggressive,
        self-interested offensive actions (e.g., repression, attempted
        seizure of city hall).
        \item During the fiscal crisis, Cosimo channeled his bank's assets
        into state debt funding, gaining the appearance of Florence's
        `financial saviour' while his opponents faced ruin.
        \item This led new men and political neutrals to delegitimize the
        oligarchs (labeling them as self-interested) and embrace Cosimo as
        saviour of the republic, leading to his triumphal recall from exile.
        \item His robust, multivocal actions gained him the ``legitimizing
        aureole of protector of the status quo'', transforming his party
        into a state.
    \end{itemize}

    \subsubsection{Conclusion}

    \noindent State centralization adn the Renaissance arose form tumultuous
historical events filtered through elite transformation. Cosimo did not
create the Medici party but shrewdly learned network rules. He used a shroud
of \textbf{multiple, impenetrable identities} to maintain robust discretion
and Solomonic legitimacy.\\

    \noindent Understanding state building requires delving into the
\textbf{relational substratum} of lives, recognizing the
\textbf{localized, ambiguous and contradictory nature} of actions, networks,
and identities. This heterogeneity explains the birth of political power.


    \section{Communities}

    \subsection{Marina G Duque. Recognizing international status: A relational approach. Inter-national Studies Quarterly, 62(3):577–592, 2018.}

    \subsubsection{Central Argument and Thesis}
    \noindent \textbf{Problem:} International relations scholarship relies
on the concept of `status' to explain phenomena like war and foreign policy,
but lacks a clear understanding of what status is and how it is achieved.\\

    \noindent \textbf{Critique of Conventional View:} Previous research
treats status as a function of a state's attributes, particularly material
capabilities like wealth and military power. This approach is a form of
material reductionism and fails to capture the social nature of status.\\

    \noindent \textbf{Author's Thesis (Relational Approach):} Status is not
derived from state attributes but from \textbf{social recognition}. It is a
relational process where a state gains admission into a `club' after being
deemed to follow it rules of membership. Status is therefore influenced by
two key social processes:
    \begin{itemize}
        \item \textbf{Self-reinforcing Dynamics:} Recognition breeds more
        recognition.
        \item \textbf{Social closure:} A state's existing relationships and
        its similarity to other states influence its ability to achieve status.
    \end{itemize}

    \subsubsection{The Conventional (Attribute-Based) Approach and Its Flaws}

    \noindent \textbf{Definition:} Defines status as a state's ranking on
valued attributes, such as economic military, and technological capabilities.\\

    \noindent \textbf{Key Flaws:}
    \begin{itemize}
        \item \textbf{Material Reductionism:} Reduces status to material
        power, making the concept analytically redundant.
        \item \textbf{Fetishism:} Mistakenly treats social relations as
        inherent properties of states. It equates status with possessing
        symbols (e.g., nuclear weapons), but these attributes have no
        intrinsic value without social agreement.
        \item \textbf{Reification:} Treats the status order as external to
        states, making status achievement an autonomous act rather than a
        social process of recognition.
        \item \textbf{Empirical Mismatch:} Fails to explain why some states
        with significant material resources receive low status (e.g., North
        Korea as a `rogue state') or why others receive more recognition
        than their capabilities would suggest (e.g., Italy, Egypt).
    \end{itemize}

\end{document}