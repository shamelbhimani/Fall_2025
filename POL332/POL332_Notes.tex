\documentclass{article}

%settings
\usepackage[utf8]{inputenc}
\usepackage[explicit]{titlesec}
\usepackage{amsmath, amsfonts, amssymb, amsthm}
\usepackage{braket}
\usepackage[margin=1.0in]{geometry}
\usepackage{bbold}
\usepackage{fancyhdr}
\usepackage{fancyvrb}
\usepackage{graphicx}
\usepackage{float}
\usepackage{longtable}
\usepackage{array}
\usepackage{hyperref}
\pagestyle{fancy}
\fancyhead[L]{\leftmark}
\fancyhead[R]{\thepage}

\title{POL332: Using Data to Understand Politics}
\author{Shamel Bhimani}
\date{Fall 2025}

\begin{document}

\maketitle

\tableofcontents
\newpage

\section{Introduction to Causality}

    \subsection{Chapter I -- Kosuke Imai. Quantitative Social Science: An Introduction. Princeton: Princeton University Press, 2017.}
    \subsubsection{Introduction to Causality}

    \noindent \textbf{Experimental Data:} examines how a treatment causally
affects and outcome by assigning varying values of the treatment variable to
different observations, and measuring their corresponding values of the
outcome.\\

    \noindent \textbf{Contingency Table:} Summarizes the relationship
between the treatment variables and the outcome variable.\\

    \noindent \textbf{Binary Variable/Dummy Variable:} Takes the value of 1
if a condition is true and 0 if the condition is false. The sample of a
binary variable equals the sample proportion of 1s. This means that the true
observations can be conveniently calculated as the \textit{sample mean}, or \textit{sample average}.\\

    To calculate the sample mean:

        \[
            \mu = \frac{\sum_{i=1}^{n} x_i}{n}
        \]
            Where:\\

$x_i$ represents each individual value or data point in the
sample;\\
    \indent $n$ represents the total number of observations or data points
in the sample.\\

\subsubsection{Causal Effects and the Counterfactual}

    \noindent Causal inference is the comparison between the factual and
the counterfactual, i.e., what actually happened and what would have
happened if a key condition were different. Unfortunately, we would never
observe this counterfactual outcome, because changing one key variable and
keeping the rest the same may, in some cases, affect internal validity.\\

    \noindent For each observation $i$, we can define the
\textbf{casual effect} of a binary treatment $T_i$ as the difference between
two potential outcomes, $Y_i(1) - Y_i(0)$, where $Y_i(1)$ represents the
outcome that would be realized under the treatment condition $(T_i = 1)$ and $Y_i(0)$ deontes the outcome that would be realized under the control condition $(T_i = 0)$.\\

    \noindent The \textbf{fundamental problem of causal inference} is that
we observe only one of the two potential outcomes, and which potential
outcome is observed depends on the treatment status. Formally, the observed
outcome $Y_i$ is equal to $Y_i(T_i)$.\\

    \noindent This simple framework of causal inference also clarifies what
is and is not an appropriate causal question. Characteristics like gender
and race, for example, are called \textit{immutable characteristics}, and
many scholars believe that causal questions about these characteristics are
not answerable. In fact, there exists a mantra which states, ``No causation
without manipulation''. However, immutable characteristics \textit{can} and
have been studied. Instead of tackling the task of directly estimating the
causal effect of race, researchers use \textit{perception scores} of the
unit of analysis.

    \subsubsection{Randomized Controlled Trials}

    \noindent In a \textbf{randomized controlled trial (RCT)}, each unit is
randomly assigned either to the treatment or control group. This
randomization of treatment assignment guarantees that the average difference
in outcome between the treatment and control groups can be attriobuted
solely to the treatment, because the two groups are on average identical to
each other in all pretreatment characteristics.\\

    \noindent \textbf{Sample Average Treatment Effect:} is defined as the
sample-average of individual-level causal effects (i.e., $Y_i(1) - Y_i(0)$).
Formally, in the potential outcomes framework:\\

    Let $Y_i(1)$ = potential outcome for unit $i$ if treated;\\
    \indent Let $Y_i(0)$ = potential outcome for unit $i$ if untreated;\\
    \indent The individual treatment effect is:\\
    \[
        \tau_{i} = Y_i(1) - Y_i(0)
    \]
    \indent The Sample Average Treatment Effect (SATE) is then:
    \[
        SATE = \frac{1}{n}\sum^{n}_{i=1}(Y_i(1)-Y_i(0))
    \]
    \indent where $n$ is the sample size.\\

    \noindent The SATE is not directly observable. For the treatment group
that received the treatment, we observe the average outcome under the
treatment but do not know what their average outcome would have been in the
absence of treatment for the same unit (the fundamental problem of causal
inference). The same problem exists for the \textit{control group} because
this group does not receive the treatment and as a result, we do not
observe the average outcome that would occur under the treatment condition.\\

    \noindent In order to estimate the average counterfactual outcome for
the treatment group, we may use the observed average outcome of the control
group. Similarly, we can use the observed average outcome of the treatment
group as an estimate of the average counterfactual outcome for the control
group. This suggests that SATE can be estimated by calculating the
difference in the average outcome between the treatment and control groups,
or the \textit{difference-in-means estimator}.










\end{document}