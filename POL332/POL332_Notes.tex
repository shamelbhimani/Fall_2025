\documentclass{article}

%settings
\usepackage[utf8]{inputenc}
\usepackage[explicit]{titlesec}
\usepackage{amsmath, amsfonts, amssymb, amsthm}
\usepackage{braket}
\usepackage[margin=1.0in]{geometry}
\usepackage{bbold}
\usepackage{fancyhdr}
\usepackage{fancyvrb}
\usepackage{graphicx}
\usepackage{float}
\usepackage{longtable}
\usepackage{array}
\usepackage{hyperref}
\pagestyle{fancy}
\fancyhead[L]{\leftmark}
\fancyhead[R]{\thepage}

\title{POL332: Using Data to Understand Politics}
\author{Shamel Bhimani}
\date{Fall 2025}

\begin{document}

\maketitle

\tableofcontents
\newpage

\section{Introduction to Causality}

    \subsection{Chapter I -- Kosuke Imai. Quantitative Social Science: An Introduction. Princeton: Princeton University Press, 2017.}
    \subsubsection{Introduction to Causality}

    \noindent \textbf{Experimental Data:} examines how a treatment causally
affects and outcome by assigning varying values of the treatment variable to
different observations, and measuring their corresponding values of the
outcome.\\

    \noindent \textbf{Contingency Table:} Summarizes the relationship
between the treatment variables and the outcome variable.\\

    \noindent \textbf{Binary Variable/Dummy Variable:} Takes the value of 1
if a condition is true and 0 if the condition is false. The sample of a
binary variable equals the sample proportion of 1s. This means that the true
observations can be conveniently calculated as the \textit{sample mean}, or \textit{sample average}.\\

    To calculate the sample mean:

        \[
            \mu = \frac{\sum_{i=1}^{n} x_i}{n}
        \]
            Where:\\

$x_i$ represents each individual value or data point in the
sample;\\
    \indent $n$ represents the total number of observations or data points
in the sample.\\

\subsubsection{Causal Effects and the Counterfactual}

    \noindent Causal inference is the comparison between the factual and
the counterfactual, i.e., what actually happened and what would have
happened if a key condition were different. Unfortunately, we would never
observe this counterfactual outcome, because changing one key variable and
keeping the rest the same may, in some cases, affect internal validity.\\

    \noindent For each observation $i$, we can define the
\textbf{casual effect} of a binary treatment $T_i$ as the difference between
two potential outcomes, $Y_i(1) - Y_i(0)$, where $Y_i(1)$ represents the
outcome that would be realized under the treatment condition $(T_i = 1)$ and $Y_i(0)$ deontes the outcome that would be realized under the control condition $(T_i = 0)$.\\

    \noindent The \textbf{fundamental problem of causal inference} is that
we observe only one of the two potential outcomes, and which potential
outcome is observed depends on the treatment status. Formally, the observed
outcome $Y_i$ is equal to $Y_i(T_i)$.\\

    \noindent This simple framework of causal inference also clarifies what
is and is not an appropriate causal question. Characteristics like gender
and race, for example, are called \textit{immutable characteristics}, and
many scholars believe that causal questions about these characteristics are
not answerable. In fact, there exists a mantra which states, ``No causation
without manipulation''. However, immutable characteristics \textit{can} and
have been studied. Instead of tackling the task of directly estimating the
causal effect of race, researchers use \textit{perception scores} of the
unit of analysis.

    \subsubsection{Randomized Controlled Trials}

    \noindent In a \textbf{randomized controlled trial (RCT)}, each unit is
randomly assigned either to the treatment or control group. This
randomization of treatment assignment guarantees that the average difference
in outcome between the treatment and control groups can be attriobuted
solely to the treatment, because the two groups are on average identical to
each other in all pretreatment characteristics.\\

    \noindent \textbf{Sample Average Treatment Effect:} is defined as the
sample-average of individual-level causal effects (i.e., $Y_i(1) - Y_i(0)$).
Formally, in the potential outcomes framework:\\

    Let $Y_i(1)$ = potential outcome for unit $i$ if treated;\\
    \indent Let $Y_i(0)$ = potential outcome for unit $i$ if untreated;\\
    \indent The individual treatment effect is:\\
    \[
        \tau_{i} = Y_i(1) - Y_i(0)
    \]
    \indent The Sample Average Treatment Effect (SATE) is then:
    \[
        SATE = \frac{1}{n}\sum^{n}_{i=1}(Y_i(1)-Y_i(0))
    \]
    \indent where $n$ is the sample size.\\

    \noindent The SATE is not directly observable. For the treatment group
that received the treatment, we observe the average outcome under the
treatment but do not know what their average outcome would have been in the
absence of treatment for the same unit (the fundamental problem of causal
inference). The same problem exists for the \textit{control group} because
this group does not receive the treatment and as a result, we do not
observe the average outcome that would occur under the treatment condition.\\

    \noindent In order to estimate the average counterfactual outcome for
the treatment group, we may use the observed average outcome of the control
group. Similarly, we can use the observed average outcome of the treatment
group as an estimate of the average counterfactual outcome for the control
group. This suggests that SATE can be estimated by calculating the
difference in the average outcome between the treatment and control groups,
or the \textit{difference-in-means estimator}.

\section{Natural Experiments}
    \subsection{Chapter II -- Kosuke Imai. Quantitative Social Science: An Introduction. Princeton: Princeton University Press, 2017.}
    \subsubsection{Observational Studies}

    \noindent Although RCTs can provide an internally valid estimate of
causal effects, in many cases social scientists are unable to randomize
treatment assignment in the real world for ethical and logistical reasons.
Here, we consider Observational Studies.\\

    \noindent \textbf{Observational Studies:} Researchers simply observe
naturally occurring events and collect and analyze the data, without direct
intervention.
    \begin{itemize}
        \item In such studies internal validity is likely to be compromised
        because of possible selection bias.
        \item External validity is often stronger than that of RCTs.
        \item Findings are more generalizable.
    \end{itemize}

    \noindent \textbf{Cross-Section Comparison Design:} More commonly known
as a \textbf{cross-sectional study}, is a type of observational research
that analyzes data from a population, or a representative subset, at a
single point in time. It is used to measure the prevalence of an outcome and
its associated factors in a specific population.\\

    \noindent The important assumption of observational studies is that the
treatment and control groups must be comparable with respect to everything
related to the outcome other than the treatment.\\

    \noindent \textbf{Confounding Variables:} A pretreatment variable that
is associated with both the treatment and the outcome variables is called a \textbf{confounder} and is a source of \textbf{confounding bias} in the estimation of the treatment effect.\\

    \noindent \textbf{Self-selection Bias:} Confounding bias due to
self-selection into the treatment group is called \textit{selection bias}.
Selection bias often arises in observational studies because researchers
have no control over who receives the treatment.
    \begin{itemize}
        \item The lack of control over treatment assignment means that those
        who self-select themselves into the treatment group may differ
        significantly from those who do not in terms of observed and
        unobserved characteristics.
        \item This makes it difficult to determine whether the observed
        difference in outcome between the treatment and control groups is
        due to the difference in the treatment condition or the differences
        in confounders.
    \end{itemize}

    \noindent In observational studies, the possibility of confounding bias
can never be ruled out. However, researchers can try to address it by means
of \textit{statistical control}.\\

    \noindent \textbf{Statistical Control:} Confounding bias can be reduced
through statistical control whereby the researcher adjusts for confounders
using statistical procedures. Some methods of statistical control are:
    \begin{itemize}
        \item \textbf{Subclassification:} The idea is to make the treatment
        and control groups as similar to each other as possible by comparing
        them within a subset of observations defined by shared values in
        pretreatment variables or a subclass.
    \end{itemize}

    \noindent In observational studies, the data collected over time are a
valuable source of information. Multiple measurements taken over time on the
same units are called \textit{longitudinal data} or \textit{panel data}.
    \begin{itemize}
        \item Longitudinal data often yield a more credible comparison of
        the treatment and control groups than \textit{cross-section data}
        because the former contain additional information about changes over
        time.
    \end{itemize}

    \noindent \textbf{Before-and-after Design:} Examines how the outcome
variable changed from the pretreatment period to the post-treatment period
for the same set of units. The design is able to adjust for any confounding
factor that is specific to each unit but does not change over time. However,
the design does not address possible bias due to time-varying confounders.\\

    \noindent \textbf{Difference-in-Differences Design:} Extends the
before-and-after design to address the confounding bias due to time treends.
The key assumption behind the DiD design is that the outcome variable
follows a parallel trend in the absence of treatment.
    \begin{itemize}
        \item Under the DiD design, the sample average causal effect
        estimate is the difference between the observed outcome after the
        treatment and the counterfactual outcome derived under the parallel
        time-trend assumption.
        \item The quantity of interest under the DiD design is called the \textit{sample average treatment efect for the treated (SATT)}.
    \end{itemize}

    \subsubsection{Sample Average Treatment Effect for the Treated (SATT)}

    \noindent The SATT is the difference between the average outcome of the
treated group with the treatment and the average outcome the sample group \textit{would have had} if they had not been treated.\\

    \noindent The formula can be expressed as:

    \[
        SATT = E[Y_{i}(1)|D_{i}=1] - E[Y_{i}(0)|D_{i}=1]
    \]
    \indent Where:\\
    \indent $Y_i(1)$ is the potential outcome for individual $i$ if
they receive the treatment;\\
    \indent $Y_i(0)$ is the potential outcome for individual $i$ if
they do not receive the treatment (the counterfactual);\\
    \indent $D_i-1$ indicates that individual $i$ is in the treatment group;\\
    \indent $E[.]$ is the expectation operator, which in this context means
we are taking the average over the individuals in the specified group.\\

    \noindent A key challenge in calculating the SATT is that we can never
observe both potential outcomes for the same individual at the same time. We
only observe the outcome for the treated group with the treatment. The
counterfactual -- what would have happened to the treated group without the
treatment -- is unobserved.\\

    \noindent Therefore, to estimate the SATT, we need to find a suitable
comparison group of untreated individuals that can serve as a proxy for
the counterfactual outcome of the treated group. A common way to express
the estimation of SATT using observed data is:\\

    \[
        SATT = \frac{1}{n_1} \sum^n_{i=1}T_i[Y_i(1)-Y_i(0)]
    \]
    \indent Where:\\
    \indent $T_i$ is the binary treatment indicator variable;\\
    \indent $n_1 = \sum^n_{i=1}T_i$ is the size of the treatment group.\\










\end{document}